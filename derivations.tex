\documentclass[12pt]{article}
\usepackage[utf8]{inputenc}
\usepackage{amsmath, amssymb}
\usepackage[
    backend=biber,
    style=numeric,
    citestyle=numeric
]{biblatex}
\addbibresource{derivations.bib}
\usepackage[breaklinks=true, linkcolor=blue, citecolor=blue, colorlinks=true]{hyperref}
% multiple eq references
\usepackage{cleverref}

\newcommand{\pluseq}{\mathrel{{+}{=}}}
\newcommand{\minuseq}{\mathrel{{-}{=}}}
\newcommand{\ns}{\ensuremath{{N_{sp}}}}
\newcommand{\conp}{constant-pressure}
\newcommand{\conv}{constant-volume}
\newcommand{\dconp}{\ensuremath{,\qquad\text{for \conp}}}
\newcommand{\dconv}{\ensuremath{,\qquad\text{for \conv}}}
\newcommand{\Ru}{\ensuremath{\mathcal{R}}}

\begin{document}
\section{Introduction}
This document contains the complete derivations for \texttt{pyJac} v2.0, which generates code to evaluate the chemical source term and analytical chemical kinetic Jacobians of \conp\slash\conv, fixed-mass, adiabatic reactors.

\section{State variables}

\begin{subequations}
\begin{align}
\Phi &= \left\{T, V, n_1, n_2 \ldots n_{\ns - 1}\right\}\dconp, \\
\Phi &= \left\{T, P, n_1, n_2 \ldots n_{\ns - 1}\right\}\dconv
\end{align}
\end{subequations}
where $T$ is the temperature, $P$ and $V$ the pressure and volume respectively, and $n_j$ the number of moles of the $j$th species in the model (containing $\ns$ total species).

From the ideal gas law, the number of moles of the final species is determined as:
\begin{subequations}
\begin{align}
n &= \frac{V P}{T \Ru} = \sum_{i=1}^{\ns}{n_i}, \label{source:moles}\\
n_{\ns} &= \frac{V P}{T \Ru} - \sum_{i=1}^{\ns - 1}{n_i}
\end{align}
\end{subequations}

\section{Chemical source terms}
The chemical source terms for this system are:
\begin{subequations}
\begin{align}
\frac{\text{d} \Phi }{\text{d} t } &= \left\{\frac{\text{d} T }{\text{d} t },\frac{\text{d} V }{\text{d} t },\frac{\text{d} n_1}{\text{d} t },\frac{\text{d} n_2 }{\text{d} t }\ldots \frac{\text{d} n_{\ns - 1} }{\text{d} t }\right\}\dconp, \\
\frac{\text{d} \Phi }{\text{d} t } &= \left\{\frac{\text{d} T }{\text{d} t },\frac{\text{d} P }{\text{d} t },\frac{\text{d} n_1}{\text{d} t },\frac{\text{d} n_2 }{\text{d} t }\ldots \frac{\text{d} n_{\ns - 1} }{\text{d} t }\right\}\dconv 
\end{align}
\end{subequations}
For both, the molar source terms are~\cite{TurnsStephenR2012Aitc}:
\begin{equation}
\frac{\text{d} n }{\text{d} t }_{k} = V \dot{\omega}_{k}
\label{source:spec}
\end{equation}
where $\dot{\omega}_k$ is the $kth$ species overall production rate, and the temperature production rates are~\cite{TurnsStephenR2012Aitc}:
\begin{subequations}
\label{source:temperature_incomplete}
\begin{align}
\frac{\text{d} T }{\text{d} t } &= - \frac{\sum_{k=1}^{\ns} H_{k} \dot{\omega}_{k}}{\sum_{k=1}^{\ns} [C]_{k} {C_{p, k}}}\dconp, \\
\frac{\text{d} T }{\text{d} t } &= - \frac{\sum_{k=1}^{\ns} U_{k} \dot{\omega}_{k}}{\sum_{k=1}^{\ns} [C]_{k} {C_{v, k}}}\dconv
\end{align}
\end{subequations}
where $H_k$, $U_k$, $C_{p,k}$ and $C_{v, k}$ are the enthalpy, internal energy, constant-pressure specific heat, constant-volume specific heat of species $k$ in molar units, while $[C]_{k}$ is the concentration.

From the ideal gas law, source terms for the volume and pressure variables may derived:
\begin{subequations}
\label{source:param_incomplete}
\begin{align}
\frac{\text{d} V }{\text{d} t } &= \frac{R_u}{P} \left(T \frac{\text{d} n }{\text{d} t } + \frac{\text{d} T }{\text{d} t } n\right)\dconp, \\
\frac{\text{d} P }{\text{d} t } &= \frac{R_u}{V} \left(T \frac{\text{d} n }{\text{d} t } + \frac{\text{d} T }{\text{d} t } n\right)\dconv
\end{align}
\end{subequations}

Finally, the conservation of mass is invoked:
\begin{equation*}
 m = \sum_{k=1}^{\ns} W_{k} n_{k},
\end{equation*}
to determine the source term of the last species:
\begin{equation}
 \frac{\text{d} n }{\text{d} t }_{\ns} = - \frac{1}{W_{\ns}} \sum_{k=1}^{\ns - 1} W_{k} \frac{\text{d} n }{\text{d} t }_{k}
 \label{source:spec_ns}
\end{equation}
where $W_{k}$ is the molecular weight of the $k$th species.
Using~\eqref{source:spec_ns}, the total molar rate of change can be determined:
\begin{align}
\frac{\text{d} n }{\text{d} t } &= \sum_{k=1}^{\ns} \frac{\text{d} n }{\text{d} t }_{k}, \nonumber \\
\frac{\text{d} n }{\text{d} t } &= \sum_{k=1}^{\ns - 1} \left(1 - \frac{W_{k}}{W_{\ns}}\right) \frac{\text{d} n }{\text{d} t }_{k}
\label{source:total_molar}
\end{align}

Combining~\cref{source:param_incomplete,source:total_molar,source:spec} gives the final form of the pressure and volume terms:
\begin{subequations}
\label{source:param_complete}
\begin{align}
\frac{\text{d} V }{\text{d} t } &= V \left(\frac{T R_u}{P} \sum_{k=1}^{-1 + \ns} \left(1 - \frac{W_{k}}{W_{\ns}}\right) \dot{\omega}_{k} + \frac{1}{T} \frac{\text{d} T }{\text{d} t }\right)\dconp, \\
\frac{\text{d} P }{\text{d} t } &= \frac{P}{T} \frac{\text{d} T }{\text{d} t } + T R_u \sum_{k=1}^{-1 + \ns} \left(1 - \frac{W_{k}}{W_{\ns}}\right) \dot{\omega}_{k}\dconv
\end{align}
\end{subequations}




\printbibliography 

\end{document}
