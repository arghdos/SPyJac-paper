\documentclass[12pt,number,sort&compress]{elsarticle}
\usepackage[utf8]{inputenc}
\usepackage[english]{babel}
\usepackage{csquotes}
\usepackage[T1]{fontenc}
\usepackage{lmodern}

% shared equations in separate file
% https://yatb.giacomodrago.com/en/post/3/latex-loading-equations-from-an-external-file.html
\usepackage{catchfilebetweentags}

\newcommand{\loadeq}[1]{%
   \ExecuteMetaData[eqn_dict.tex]{eq#1}%
}

% dcases
\usepackage{mathtools}
% for text right arrow
\usepackage{textcomp}
\usepackage{amssymb}
\usepackage{amsmath,eqparbox,xparse}
\usepackage[breaklinks=true, linkcolor=blue, citecolor=blue, colorlinks=true]{hyperref}
% load after hyperref
\usepackage[english,capitalise]{cleveref}
\bibliographystyle{elsarticle-num}

% https://tex.stackexchange.com/a/34412/5764
\makeatletter
\NewDocumentCommand{\eqmathbox}{o O{c} m}{%
  \IfValueTF{#1}
    {\def\eqmathbox@##1##2{\eqmakebox[#1][#2]{$##1##2$}}}
    {\def\eqmathbox@##1##2{\eqmakebox{$##1##2$}}}
  \mathpalette\eqmathbox@{#3}
}
\makeatother

\newcommand{\pluseq}{\mathrel{{+}{=}}}
\newcommand{\minuseq}{\mathrel{{-}{=}}}
\newcommand{\ns}{N_{sp}}
\newcommand{\nr}{N_{reac}}
\newcommand{\conp}{CONP}
\newcommand{\conv}{CONV}
\newcommand{\dconp}{\ifmmode\text{for \conp,}\else for \conp,\fi}
\newcommand{\dconv}{\ifmmode\text{for \conv,}\else for \conv.\fi}
\newcommand{\Ru}{\mathcal{R}}

% simple command for equal-sized eqmathbox's on the lhs / rhs and conditional
% arguements are tag, text, alignment
\NewDocumentCommand{\lhs}{mmO{r}}{\eqmathbox[LHS_#1][#3]{#2}}
\NewDocumentCommand{\rhs}{mmO{l}}{\eqmathbox[RHS_#1][#3]{#2}}
\NewDocumentCommand{\cond}{mmO{r}}{\eqmakebox[COND_#1][#3]{#2}}
\NewDocumentCommand{\mathcond}{mmO{r}}{\eqmathbox[COND_#1][#3]{#2}}

% consistent multiline equation spacing
\NewDocumentCommand{\mathindent}{O{space}}{\hphantom{\mathrel{#1}}}

% bigger parenthesis
% https://tex.stackexchange.com/questions/6794/about-big-parenthesis-larger-than-bigg
\makeatletter
\newcommand{\vast}{\bBigg@{4}}
\newcommand{\Vast}{\bBigg@{5}}
\makeatother

% treat subequations as equation for cleverref
% https://tex.stackexchange.com/a/308627/56227
\crefalias{subequation}{equation}


%fix to dcases from here:http://tex.stackexchange.com/questions/252410/centering-in-dcases-environment/252414
\MHInternalSyntaxOn
\renewcommand{\dcases}
 {
  \MT_start_cases:nnnn
    {\quad}
    {$\m@th\displaystyle##$\hfil}
    {$\m@th\displaystyle##$\hfil}
    {\lbrace}
 }
\MHInternalSyntaxOff

% allow page breaks for long equations:
% https://tex.stackexchange.com/questions/51682/is-it-possible-to-pagebreak-aligned-equations
\allowdisplaybreaks

\begin{document}
\section{Introduction}
This document contains the derivations for \texttt{pyJac} v2.0, which generates code to evaluate the chemical source terms and analytical chemical kinetic Jacobians of constant-pressure\slash constant-volume, fixed-mass, adiabatic reactors.
Note that equations specific to the constant-pressure or constant-volume derivation will be marked with \conp~or \conv~respectively.
The derivations in this document based upon the output of the script \href{https://github.com/arghdos/SPyJac-paper/blob/master/derivations/scripts/derivations.py}{derivations.py} in the GitHub repository for this paper.
This script depends on the symbolic mathematics library SymPy~\cite{sympy}, and was tested with versions 1.0 and 1.1.1

\section{Governing equations}
\subsection{State variables}
The state vector for this derivation consists of the temperature, a thermodynamic state parameter (pressure or volume) and the number of moles of all species except the last in the model:
\loadeq{state}
where $T$ is the temperature, $V$ and $P$ the volume and pressure respectively, and $n_j$ the number of moles of the $j$th species in the model (containing $\ns$ total species).

From the ideal gas law, the number of moles of the final species is determined as:
\begin{align}
n &= \frac{V P}{T \Ru} = \sum_{i=1}^{\ns}{n_i}\; , \nonumber \\
n_{\ns} &= \frac{V P}{T \Ru} - \sum_{i=1}^{\ns - 1}{n_i}
\end{align}

\subsection{Thermo-chemical source terms}
The evolution of the thermo-chemical state variables of this system is described by a set of ordinary-differential equations:
\loadeq{dstate}

For both, the molar source terms are~\cite{TurnsStephenR2012Aitc}:
\loadeq{dmolar}
where $\dot{\omega}_k$ is the $kth$ species' overall production rate, and the temperature production rate~\cite{TurnsStephenR2012Aitc} is:
\loadeq{dtemp}
where $H_k$, $U_k$, $C_{p,k}$ and $C_{v, k}$ are the enthalpy, internal energy, constant-pressure and constant-volume specific heat of species $k$ in molar units respectively, while $[C]_{k}$ is the concentration, given by:
\begin{equation}
 [C]_{k} = \frac{n_{k}}{V}
\end{equation}

From the ideal gas law, source terms for the volume and pressure variables may derived:
\begin{subequations}
\label{e:param_incomplete}
\begin{align}
\frac{\text{d} V }{\text{d} t } &= \frac{\Ru}{P} \left(T \frac{\text{d} n }{\text{d} t } + \frac{\text{d} T }{\text{d} t } n\right) & \text{\dconp} \\
\frac{\text{d} P }{\text{d} t } &= \frac{\Ru}{V} \left(T \frac{\text{d} n }{\text{d} t } + \frac{\text{d} T }{\text{d} t } n\right) & \text{\dconv}
\end{align}
\end{subequations}

To determine $\frac{\text{d} n }{\text{d} t }$, conservation of mass is invoked:
\begin{equation}
 \label{e:mass_conservation}
 \frac{\text{d} m }{\text{d} t } = 0 = \sum_{k=1}^{\ns}  W_{k} \frac{\text{d} n_{k} }{\text{d} t },
\end{equation}
where $W_{k}$ is the molecular weight of the $k$th species.
From~\cref{e:mass_conservation}, the source term of the last species may be written in terms of the species included in the state vector:
\begin{equation}
 \frac{\text{d} n }{\text{d} t }_{\ns} = \frac{-1}{W_{\ns}} \sum_{k=1}^{\ns - 1} W_{k} \frac{\text{d} n_{k} }{\text{d} t }
 \label{e:spec_ns}
\end{equation}
Using~\cref{e:spec_ns}, the total molar rate of change can be determined:
\begin{align}
\frac{\text{d} n }{\text{d} t } &= \sum_{k=1}^{\ns} \frac{\text{d} n_{k} }{\text{d} t }, \nonumber \\
\frac{\text{d} n }{\text{d} t } &= \sum_{k=1}^{\ns - 1} \left(1 - \frac{W_{k}}{W_{\ns}}\right) \frac{\text{d} n_{k} }{\text{d} t }
\label{e:total_molar}
\end{align}

Combining~\cref{e:spec,e:param_incomplete,e:total_molar} gives the final form of the pressure and volume terms:
\loadeq{dparam}

Additionally, the concentration and moles of the last species in the model (which is not a state variable) may be expanded in terms of the state variables of the system:
\begin{equation}
 \label{e:last_spec_conc}
   [C]_{\ns} = [C] - \sum_{k=1}^{\ns  - 1} [C]_{k}
\end{equation}
where $[C]$ is the total concentration:
\begin{equation}
 [C] = \frac{P}{T \Ru}
\end{equation}

Using~\cref{e:last_spec_conc,e:spec,e:spec_ns}, the temperature source terms may be expanded to contain only state variables (i.e., by removal of the last species' source term and overall production rate):
\begin{subequations}
\label{e:temperature_complete}
\begin{align}
\frac{\text{d} T }{\text{d} t } &= - \frac{\sum_{k=1}^{\ns  - 1} \left(H_{k} - \frac{W_{k} H_{\ns}}{W_{\ns}}\right) \dot{\omega}_{k}}{[C] {C_{p,\ns}} + \sum_{k=1}^{\ns  - 1} \left({C_{p, k}} - {C_{p,\ns}}\right) [C]_{k}} &\text{\dconp}\\
\frac{\text{d} T }{\text{d} t } &= - \frac{\sum_{k=1}^{\ns  - 1} \left(U_{k} - \frac{W_{k} U_{\ns}}{W_{\ns}}\right) \dot{\omega}_{k}}{[C] {C_{v,\ns}} + \sum_{k=1}^{\ns  - 1} \left({C_{v, k}} - {C_{v,\ns}}\right) [C]_{k}} &\text{\dconv}
\end{align}
\end{subequations}
This form of the temperature source term will be used for derivation of Jacobian terms, but~\cref{e:temperature_incomplete} will often be used as well due to its compactness.

\subsection{Thermal properties}
The standard-state thermodynamic properties (in molar units) for a gaseous species $k$ is defined using the standard seven-coefficient polynomial of Gordon and McBride~\cite{gordon1994computer}:
\begin{align}
\frac{C_{p,k}^{\circ}}{\mathcal{R}} &= a_{0,k} + T \left( a_{1,k} + T \left( a_{2,k} + T \left( a_{3,k} + a_{4,k} T \right) \right) \right) \label{e:cpk} \\
\frac{H_k^{\circ}}{\mathcal{R}} &= T \left( a_{0,k} + T \left( \frac{a_{1,k}}{2} + T \left( \frac{a_{2,k}}{3} + T \left( \frac{a_{3,k}}{4} + \frac{a_{4,k}}{5} T \right) \right) \right) \right) + a_{5,k} \label{e:hk} \\
\frac{S_k^{\circ}}{\mathcal{R}} &= a_{0,k} \ln T + T \left( a_{1,k} + T \left( \frac{a_{2,k}}{2} + T \left( \frac{a_{3,k}}{3} + \frac{a_{4,k}}{4} T \right) \right) \right) + a_{6,k} \label{e:sk}
\end{align}
where $C_{p,k}$ and $H_k$ are as described previously, $S_k$ is the entropy in molar units, and the ${}^{\circ}$ indicates a standard-state property at one atmosphere (equivalent to the property at any pressure for calorically perfect gases).

\subsection{Reaction rate expressions}
The net species production rate for species $k$ is defined as:
\begin{equation}
 \label{e:spec_rop}
 \dot{\omega}_{k} = \sum_{i=1}^{\nr} \nu_{k,i} q_{i}
\end{equation}
where $\nr$ is the number of reactions in the chemical kinetic model, $\nu_{k, i}$ the overall stoichiometric coefficient of species $k$ in reaction $i$ and $q_i$ the net rate-of-progress for reaction $i$:
\begin{align}
\nu_{k,i} &= \nu^{\prime\prime}_{k,i} - \nu^{\prime}_{k,i} \\
q_{i} &= R_{i} c_{i}
\end{align}
with $\nu^{\prime}_{k,i}$ and $\nu^{\prime\prime}_{k,i}$ the product and reactant stoichiometric cofficients (respectively) of species $k$ in reaction $i$.
The base rate-of-progress for the $i$th reversible reaction $R_{i}$ is given by:
\begin{align}
R_{i} &= {R_{f,i}} - {R_{r,i}} \label{e:ropnet}\\
{R_{f, i}} &= {k_{f, i}} \prod_{k=1}^{\ns} [C]_{k}^{\nu^{\prime}_{k,i}} \label{e:ropf}\\
{R_{r, i}} &= {k_{r, i}} \prod_{k=1}^{\ns} [C]_{k}^{\nu^{\prime\prime}_{k,i}} \label{e:ropr}
\end{align}
where ${k_{f, i}}$ and ${k_{r, i}}$ are the forward and reverse reaction coefficients (respectively) for the $i$th reaction, and the third-body\slash pressure modification $c_{i}$ is given by:
\begin{equation}
\label{e:rxn_pressure}
c_i = \begin{dcases}
  \rhs{pmod}{1}[c] &\text{for elementary reactions,} \\
  \rhs{pmod}{[X]_i}[c] &\text{for third-body enhanced reactions,} \\
  \rhs{pmod}{\frac{P_{r,i}}{1 + P_{r,i}} F_i} &\text{for unimolecular\slash~recombination fall-off reactions, and} \\
  \rhs{pmod}{\frac{1}{1 + P_{r,i}} F_i} &\text{for chemically-activated bimolecular reactions,}
  \end{dcases}
\end{equation}
where for the $i$th reaction $[X]_i$ is the third-body concentration, $P_{r,i}$ is the reduced pressure, and $F_i$ is the falloff blending factor.
These terms are defined in the following sections.

The forward reaction rate coefficient $k_{f, i}$ is given by the three-parameter Arrhenius expression:
\begin{equation}
  \label{e:arrhenius}
  {k_{f, i}} = A_{i} T^{\beta_{i}} \operatorname{exp}\left({- \frac{{E_{a}}_{i}}{T \Ru}}\right) \;,
\end{equation}
where $A_i$ is the pre-exponential factor, $\beta_i$ is the temperature exponent, and $T_{a, i}$ is the activation temperature given by $T_{a, i} = E_{a, i} / \mathcal{R}$.

As given by Lu and Law~\cite{Lu:2009gh}, depending on the value of the Arrhenius parameters, $k_{f,i}$ can be calculated in different ways to minimize the computational cost:
\begin{equation}
  k_{f,i} =
  \begin{dcases}
  A_i & \text{if } \beta = 0 \text{ and } T_{a,i} = 0 \;, \\
  \exp \left( \log A_i + \beta_i \log T \right)   & \text{if } \beta_i \neq 0 \text{ and } T_{a, i} = 0 \;, \\
  \exp \left( \log A_i + \beta_i \log T - T_{a, i} / T \right) & \text{if } \beta_i \neq 0 \text{ and } T_{a, i} \neq 0 \;, \\
  \exp \left( \log A_i - T_{a, i} / T \right)  & \text{if } \beta_i = 0 \text{ and } T_{a, i} \neq 0 \;, \text{ and} \\
  A_i \prod^{\beta_i} T & \text{if } T_{a, i} = 0 \text{ and } \beta_i \in \mathbb{Z} \;,
  \end{dcases}
\end{equation}
where $\mathbb{Z}$ is the set of integers; the extent of this specialization can be controlled when generating code via \texttt{pyjac}.

\subsection{Reverse rate coefficient}
By definition, the reverse rate coefficient, ${k_{r, i}}$, of irreversible reactions is zero, while reversible reactions may have non-zero ${k_{r, i}}$.
Note that in \texttt{pyJac}, reversible reactions with explicit reverse Arrhenius parameters are split into two irreversible reactions; this simplifies calculation inside the generated code, and eases comparison to Cantera~\cite{Goodwin:2015aa} which applies the same transformation.
For reversible reactions without an explicit parameterization, the reverse rate coefficient is calculated from ratio of the forward rate coefficient and the equilibrium constant:
\begin{align}
 {k_{r, i}} &= \frac{{k_{f, i}}}{{K_{c, i}}}\; , \label{e:kr}\\
 {K_{c, i}} &= {K_{p,i}}\left(\frac{P_{atm}}{T \Ru}\right)^{\sum_{k=1}^{\ns} \nu_{k,i}} \; ,\text{ and} \label{e:kc}\\
 {K_{p,i}} &= \text{exp}\left(\frac{\Delta S^{\circ}_i}{\Ru} - \frac{\Delta H^{\circ}_i}{\Ru T}\right) = \text{exp}\left(\sum_{k=1}^{\ns}\nu_{ki}\left(\frac{S^{\circ}_k}{\Ru} - \frac{H^{\circ}_k}{\Ru T}\right)\right) \label{e:kp}
\end{align}
where $P_{atm}$ is the pressure of one standard atmosphere in appropriate units.

By combining~\cref{e:kc,e:kp}, we obtain:
\begin{equation}
 \label{e:kc_in_kp}
 {K_{c, i}} = \left(\left(\frac{P_{atm}}{\Ru}\right)^{\sum_{k=1}^{\ns} \nu_{k,i}}\right) \operatorname{exp}\left({\sum_{k=1}^{\ns} \nu_{k,i} B_{k}}\right)
\end{equation}
where, $B_k$ is:
\begin{align}
 \label{e:Bk}
 B_{k} &= \frac{S^{\circ}_k}{\Ru} - \frac{H^{\circ}_k}{\Ru T} - \ln{T} \nonumber\; , \\
 B_{k} &= T \left(T \left(T \left(\frac{T a_{k,4}}{20} + \frac{a_{k,3}}{12}\right) + \frac{a_{k,2}}{6}\right) + \frac{a_{k,1}}{2}\right) \nonumber \\
       & \quad + \left(a_{k,0} - 1\right) \log{\left (T \right )} - a_{k,0} + a_{k,6} - \frac{a_{k,5}}{T}
\end{align}
from~\cref{e:hk,e:sk}.

\subsection{Third-body effects}
\label{s:thdbody}

For a reaction enhanced (or diminished) by the presence of a third body, the reaction rate is modified by the third-body concentration $[X]_i$ given by
\begin{equation}
[X]_{i} = \sum_{k=1}^{\ns} \alpha_{k,i} [C]_{k} \;,
\end{equation}
where $\alpha_{k,i}$ is the third-body efficiency of species $k$ in the $i$th reaction.
For a default third-body efficiency of $\alpha_{k,i} = 1$, this may be rearranged to the compact-storage form:
\begin{equation}
 [X]_{i} = [C] + \sum_{k=1}^{\ns} \left(\alpha_{k,i} - 1\right) [C]_{k} \;,
\end{equation}
where only species with non-default third-body efficiencies must be stored in the model.
Expanding the concentration of the last species gives:
\begin{equation}
\label{e:thd_mix}
 [X]_{i}=[C] \alpha_{\ns,i} + \sum_{k=1}^{\ns  - 1} \left(- \alpha_{\ns,i} + \alpha_{k,i}\right) [C]_{k}\;.
\end{equation}
This form will be denoted as the mixture-based (or \textbf{mix} for short) third-body efficiency.

If all species in the mixture contribute equally as third bodies, then $\alpha_{k,i} = 1$ for all species:
\begin{equation}
\label{e:thd_unity}
 [X]_{i} = [C] = \frac{P}{\Ru T} \;.
\end{equation}
This form will be called the \textbf{unity}-based third-body efficiency.

In addition, a single species may act as the third body in which case
\begin{equation}
 [X]_{i} = [C_m]
\end{equation}
where the $m$th species is the third body.
For evaluation of Jacobian entries, the expanded form:
\begin{equation}
\label{e:thd_spec}
 [X]_{i}=\left([C] - \sum_{k=1}^{\ns  - 1} [C]_{k}\right) \delta_{\ns m} + \left(- \delta_{\ns m} + 1\right) [C]_{m}
\end{equation}
will be used, where the Kronecker delta $\delta_{\ns m}$ is unity if and only if the third body species $m$ is the last species in the model.
\Cref{e:thd_spec} will be referred to as the \textbf{species}-based third-body efficiency.


\subsection{Falloff reactions}
Unlike elementary and third-body reactions, falloff reactions exhibit a pressure dependence described as a blending of rates at low- and high-pressure limits; thus, the rate coefficients depend on a mixture of low-pressure ($k_{0, i}$) and high-pressure-limit ($k_{\infty,i}$) coefficients, each with corresponding Arrhenius parameters and expressed using \cref{e:arrhenius}.
The ratio of the coefficients $k_{0, i}$ and $k_{\infty, i}$, combined with the third-body concentration, define a reduced pressure $P_{r,i}$ given by
\begin{equation}
 \label{e:pr}
 P_{r, i}=\frac{[X]_{i} k_{0, i}}{k_{\infty, i}}
\end{equation}
where $[X]_{i}$ is the appropriate third-body concentration as described in~\cref{s:thdbody}.

The falloff blending factor $F_i$ used in \cref{e:rxn_pressure} is determined based on one of three representations: the Lindemann~\cite{Lindemann:1922cz}, Troe~\cite{Gilbert:1983bb}, and SRI~\cite{Stewart:1989gj} falloff approaches
\begin{equation}
\label{e:fi}
F_i = \begin{dcases}
1 &\text{for Lindemann,} \\
F_{\text{cent}}^{ \left( 1 + ( A_{\text{Troe}} / B_{\text{Troe}} )^2 \right)^{-1} } &\text{for Troe, or} \\ 
d T^e \left( a \cdot \exp \left( -\frac{b}{T} \right) + \exp \left( -\frac{T}{c} \right) \right)^X &\text{for SRI.} 
\end{dcases}
\end{equation}
The Troe representation is described by the variables:
\begin{align}
 F_{cent} &= a \operatorname{exp}\left({- \frac{T}{T^{*}}}\right) + \left(- a + 1\right) \operatorname{exp}\left({- \frac{T}{T^{***}}}\right) + \operatorname{exp}\left({- \frac{T^{**}}{T}}\right) \;, \\
 A_{Troe} &= - 0.67 \log_{10}{\left (F_{cent} \right )} + \log_{10}{\left (P_{r, i} \right )} - 0.4 \;\text{, and}\\
 B_{Troe} &= - 1.1762 \log_{10}{\left (F_{cent} \right )} - 0.14 \log_{10}{\left (P_{r, i} \right )} + 0.806
\end{align}
where $a$, $T^{***}$, $T^*$, and $T^{**}$ are specified parameters.
The final parameter $T^{**}$ is optional, and, if it is not used, the final term of $F_{\text{cent}}$ is omitted.

The exponent used in the SRI representation is given by:
\begin{equation}
 X = \frac{1}{\log_{10}^{2}{\left (P_{r, i} \right )} + 1}
\end{equation}
The parameters $a$, $b$, and $c$ in the SRI falloff blending factor are required while $d$ and $e$ are optional; if not specified, $d = 1$ and $e = 0$.

\subsection{Pressure-dependent reactions}
\label{s:pdep}

In addition to the falloff approach given previously, two additional formulations can be used to describe the pressure dependence of reactions that do not follow the modification factor $c_i$ approach.
The first involves logarithmic interpolation between Arrhenius rates at two pressures~\cite{chemkin:2012,Goodwin:2015aa} (often termed P-Log reactions), each evaluated using~\cref{e:arrhenius}:
\begin{align}
k_1 (T) &= A_1 T^{\beta_1} \exp \left( -\frac{T_{a, 1}}{T} \right) \text{ at } P_1 \text{ and} \label{e:plog_k1} \\
k_2 (T) &= A_2 T^{\beta_2} \exp \left( -\frac{T_{a, 2}}{T} \right) \text{ at } P_2 \;, \label{e:plog_k2}
\end{align}
where the Arrhenius coefficients are given for each pressure $p_1$ and $p_2$.
Then, the reaction rate coefficient at a particular pressure $p$ between $p_1$ and $p_2$ can be determined through logarithmic interpolation:
\begin{equation}
\label{e:kf_plog}
\log{\left ({k_{f, i}} \right )} = \frac{\left(- \log{\left (k_{1} \right )} + \log{\left (k_{2} \right )}\right) \left(- \log{\left (P_{1} \right )} + \log{\left (P \right )}\right)}{- \log{\left (P_{1} \right )} + \log{\left (P_{2} \right )}} + \log{\left (k_{1} \right )}
\end{equation}

In addition, the pressure dependence of a reaction can be expressed through a bivariate Chebyshev polynomial fit~\cite{Venkatesh:1997hv,Venkatesh:1997ik,Venkatesh:2000gj,chemkin:2012,Goodwin:2015aa}:
\begin{equation}
\label{e:kf_cheb}
\log_{10} k_f (T, p) = \sum_{i = 1}^{N_T} \sum_{j = 1}^{N_p} \eta_{ij} \phi_i (\tilde{T}) \phi_j \left(\tilde{p}\right) \;,
\end{equation}
where $\eta_{ij}$ is the coefficient corresponding to the grid points $i$ and $j$, $N_T$ and $N_p$ are the numbers of grid points for temperature and pressure, respectively, and $\phi_n$ is the Chebyshev polynomial of the first kind of degree $n - 1$ typically expressed as
\begin{equation}
\phi_n (x) = \mathcal{T}_{n-1} (x) = \cos \left( (n - 1) \arccos (x) \right) \quad \text{for } |x| \leq 1 \;.
\end{equation}
The reduced temperature $\tilde{T}$ and pressure $\tilde{p}$ are given by
\begin{align}
\tilde{T} &\equiv \frac{2 T^{-1} - T^{-1}_{\min} - T^{-1}_{\max}}{T^{-1}_{\max} - T^{-1}_{\min}} \quad\text{and} \\
\tilde{P} &\equiv \frac{2\log_{10} P - \log_{10} P_{\min} - \log_{10} P_{\max}}{\log_{10} P_{\max} - \log_{10} P_{\min}} \;,
\end{align}
where $T_{\min} \leq T \leq T_{\max}$ and $p_{\min} \leq p \leq p_{\max}$ describe the ranges of validity for temperature and pressure.

\section{Jacobian derivation}

Let $\mathcal{J}$ denote the Jacobian matrix corresponding to the set of ODEs defined in~\cref{e:source_terms}.
$\mathcal{J}$ is filled with the partial derivatives $\partial f / \partial \Phi$, such that:
\begin{equation}
 \label{e:jac_general}
 \mathcal{J}_{i,j} = \frac{\partial f_i}{\partial \Phi_j},\qquad j=1 \ldots \ns - 1
\end{equation}
where $i$ and $j$ correspond to the row and column, respectively, of the entry in the Jacobian matrix.
The Jacobian entries resulting from~\cref{e:jac_general} are derived in~\cref{s:energy_derivs,s:molar_source_derivatives,s:parameter_source_derivatives}, while various subcomponents of the Jacobian are derived in~\cref{s:dri_dt,s:dri_dnj,s:dri_de,s:dri_pdep,s:dpmod_fall}.
The final form of the Jacobian can be found in~\cref{s:jac_final}, which provides a useful summary for this document.

We note that some parts of the Jacobian derivation---in particular derivatives of the energy equation---are quite complicated.
The aim of this document is to provide the reader with an overview of the derivation process; some intermediate steps are left out due to brevity and typesetting constraints.
For a complete derivation, the reader is directed to the output of the \href{https://github.com/arghdos/SPyJac-paper/blob/master/derivations/scripts/derivations.py}{derivations.py} script, from which this document was compiled.

\subsection{Temperature source term derivatives}
\label{s:energy_derivs}
\subsubsection{Temperature derivative}
\label{s:denergy_dt}
First, we will derive the term:
\begin{align}
 \mathcal{J}_{1,1} &= \frac{\partial}{\partial T} \frac{\text{d} T}{\text{d} t} \nonumber \\
 \intertext{or more simply:}
 \mathcal{J}_{1,1} &= \frac{\partial\dot{T}}{\partial{T}}
\end{align}
To do so, we will more closely follow the \conp~energy equation derivative with respect to temperature as the process is largely identical between \conp~and \conv.

The derivative of a species concentration with respect to temperature for both \conp~and \conv~is:
\begin{align}
 \frac{\partial [C]_{k}}{\partial T} = 
 \begin{dcases}
  \frac{\partial}{\partial T} \left(\frac{n_k}{V}\right) = 0 & \left\{k \ne \ns\right\} \\
  \frac{\partial}{\partial T} \left([C] - \sum_{k=1}^{\ns - 1}{\frac{n_k}{V}}\right) = -\frac{[C]}{T} & \left\{k = \ns\right\}
 \end{dcases}
\end{align}
as the pressure and volume are either constants (for \conp~and \conv~respectively) or a state variable and a function of time only.

Next, the derivative of full form of the energy source term~\cref{e:temperature_complete} with respect to temperature will be considered.
To make the result more managable, the denominator of~\cref{e:temperature_complete} is first collapsed back into the form of~\cref{e:temperature_incomplete}, yielding for~\conp:
\begin{align}
 \frac{\partial\dot{T}}{\partial{T}} &= - \frac{\left( [C] \left( \frac{{C_p}_{\ns}}{T} - \frac{\text{d} {C_p} }{\text{d} T }_{\ns} \right) - \sum_{k=1}^{-1 + \ns} \left(- \frac{\text{d} {C_p} }{\text{d} T }_{\ns} + \frac{\text{d} {C_p} }{\text{d} T }_{k}\right) [C]_{k}\right)}{\left(\sum_{k=1}^{\ns} [C]_{k} {C_p}_{k}\right)^{2}} \times \nonumber \\
				     & \mathindent[largerindent] \sum_{k=1}^{-1 + \ns} \left(H_{k} - \frac{W_{k} H_{\ns}}{W_{\ns}}\right) \dot{\omega}_{k} - \nonumber \\
				     & \mathindent \frac{\sum_{k=1}^{-1 + \ns} \left(\left(H_{k} - \frac{W_{k} H_{\ns}}{W_{\ns}}\right) \frac{\partial \dot{\omega} }{\partial T }_{k} + \left(\frac{\text{d} H }{\text{d} T }_{k} - \frac{W_{k}}{W_{\ns}} \frac{\text{d} H }{\text{d} T }_{\ns}\right) \dot{\omega}_{k}\right)}{\sum_{k=1}^{\ns} [C]_{k} {C_p}_{k}}
\end{align}
Next, we may recognize that~\cref{e:temperature_incomplete} may be factored out of the first terms, yielding:
\begin{align}
 \frac{\partial\dot{T}}{\partial{T}} &= \frac{1}{\sum_{k=1}^{\ns} [C]_{k} {C_p}_{k}} \Biggl[ \frac{\text{d} T }{\text{d} t } \times \nonumber \\
				     & \left( [C] \left( \frac{{C_p}_{\ns}}{T} - \frac{\text{d} {C_p} }{\text{d} T }_{\ns} \right) - \sum_{k=1}^{-1 + \ns} \left(- \frac{\text{d} {C_p} }{\text{d} T }_{\ns} + \frac{\text{d} {C_p} }{\text{d} T }_{k}\right) [C]_{k}\right) - \nonumber \\
				     & \sum_{k=1}^{-1 + \ns} \left(\left(H_{k} - \frac{W_{k} H_{\ns}}{W_{\ns}}\right) \frac{\partial \dot{\omega} }{\partial T }_{k} + \left(\frac{\text{d} H }{\text{d} T }_{k} - \frac{W_{k}}{W_{\ns}} \frac{\text{d} H }{\text{d} T }_{\ns}\right) \dot{\omega}_{k}\right) \Biggr]
\end{align}
Now, we invoke the identity:
\begin{equation}
 [C] = \sum_{k=1}^{\ns} [C]_{k}
\end{equation}
to simplify the specific heat terms:
\begin{align}
 \frac{\partial\dot{T}}{\partial{T}} &= \frac{1}{\sum_{k=1}^{\ns} [C]_{k} {C_p}_{k}} \Biggl[ \frac{\text{d} T }{\text{d} t } \sum_{k=1}^{\ns} \left(- \frac{\text{d} {C_p} }{\text{d} T }_{k} + \frac{{C_p}_{\ns}}{T}\right) [C]_{k} + \nonumber \\
				     & \sum_{k=1}^{-1 + \ns} \biggl(\left(- H_{k} + \frac{W_{k} H_{\ns}}{W_{\ns}}\right) \frac{\partial \dot{\omega} }{\partial T }_{k} + \nonumber \\
				     & \mathindent[larger] \left(- \frac{\text{d} H }{\text{d} T }_{k} + \frac{W_{k}}{W_{\ns}} \frac{\text{d} H }{\text{d} T }_{\ns}\right) \dot{\omega}_{k}\biggr)\Biggr]
\end{align}
and finally, we may substitute in the specific heat for the temperature derivative of enthalpy, giving:
\loadeq{dTdotdT}
The species production rate derivative with respect to temperature will be defined in~\cref{s:d_spec_prod}.

\subsubsection{Molar derivative}
Next the molar derivatives of the energy equation:
\begin{align}
 \mathcal{J}_{1,j} = \frac{\partial\dot{T}}{\partial{n_j}}
\end{align}

First, the derivative of a species concentration with respect to the moles of species $j$ is computed:
\begin{align}
 \frac{\partial [C_k]}{\partial n_j} &=
 \begin{dcases}
 \frac{\delta_{j k}}{V} & k \ne \ns \\
 -\frac{1}{V} & k = \ns
 \end{dcases} \\
\intertext{Or put succinctly:}
\label{e:dck_dnj}
\frac{\partial [C_k]}{\partial n_j} &= \frac{\delta_{j k} - \delta_{\ns k}}{V}
\end{align}

Similar to~\cref{s:denergy_dt} the derivative of the energy equation with respect to the moles of species $j$---after collapsing the specific-heat\slash~concentration sum---are given by:
\begin{align}
 \frac{\partial\dot{T}}{\partial{n_j}} &= \frac{\left(\sum_{k=1}^{-1 + \ns} \left(H_{k} - \frac{W_{k} H_{\ns}}{W_{\ns}}\right) \dot{\omega}_{k}\right) \sum_{k=1}^{-1 + \ns} - \frac{\delta_{j k}}{V} \left({C_p}_{\ns} - {C_p}_{k}\right)}{\left(\sum_{k=1}^{\ns} [C]_{k} {C_p}_{k}\right)^{2}} - \nonumber \\ 
				       & \mathindent \frac{1}{\sum_{k=1}^{\ns} [C]_{k} {C_p}_{k}} \sum_{k=1}^{-1 + \ns} \left(H_{k} - \frac{W_{k} H_{\ns}}{W_{\ns}}\right) \frac{\partial \dot{\omega} }{\partial {n_j} }_{k}
\end{align}
Next, the Kronecker delta summation is simplified, and the the compact form of the energy equation factored out yielding:
\begin{subequations}
\begin{align}
 \frac{\partial\dot{T}}{\partial{n_j}} &= \frac{1}{\sum_{k=1}^{\ns} [C]_{k} {C_p}_{k}} \Biggl[- \sum_{k=1}^{-1 + \ns} \left(H_{k} - \frac{W_{k} H_{\ns}}{W_{\ns}}\right) \frac{\partial \dot{\omega} }{\partial {n_j} }_{k} & \nonumber \\
				       &  \mathindent[largerindentand] - \frac{1}{V} \frac{\text{d} T }{\text{d} t } \left(- {C_p}_{\ns} + {C_p}_{j}\right)\Biggr] & \text{\dconp} \\
\intertext{and:}
 \frac{\partial\dot{T}}{\partial{n_j}} &= \frac{1}{\sum_{k=1}^{\ns} [C]_{k} {C_v}_{k}} \Biggl[- \sum_{k=1}^{-1 + \ns} \left(U_{k} - \frac{W_{k} U_{\ns}}{W_{\ns}}\right) \frac{\partial \dot{\omega} }{\partial {n_j} }_{k} & \nonumber \\ 
				       & \mathindent[largerindentand] - \frac{1}{V} \frac{\text{d} T }{\text{d} t } \left(- {C_v}_{\ns} + {C_v}_{j}\right) \Biggr] & \text{\dconv}
\end{align}
\end{subequations}
Again, the species production rate derivative with respect to moles will be explored in~\cref{s:d_spec_prod}.

\subsubsection{State parameter derivatives}
Next, the derivative of the energy equation with respect to the thermodynamic state parameter will be considered:
\begin{subequations}
\begin{align}
 \mathcal{J}_{1, 2} & = \frac{\partial\dot{T}}{\partial{V}} & \text{\dconp} \\
 \mathcal{J}_{1, 2} & = \frac{\partial\dot{T}}{\partial{P}} & \text{\dconv}
\end{align}
\end{subequations}

First, the derivative of the concentration of a species $k$ with respect to the state parameter is:
\begin{subequations}
 \label{e:dck_de}
 \begin{align}
 \lhs{dck}{\frac{\partial [C]_{k} }{\partial V }} &=
  \begin{dcases}
  \rhs{dck}{-\frac{[C]_{k}}{V}} & \mathcond{dck}{k \ne \ns} \\
  \rhs{dck}{\frac{1}{V} \sum_{k=1}^{\ns  - 1} [C]_{k}} & \mathcond{dck}{k = \ns}
 \end{dcases}&\cond{dck}{\dconp} \\
 \lhs{dck}{\frac{\partial [C]_{k} }{\partial P }} &=
 \begin{dcases}
   \rhs{dck}{\mathindent[s]0} & \mathcond{dck}{k \ne \ns} \\
   \rhs{dck}{\frac{1}{T \Ru}} & \mathcond{dck}{k = \ns}
 \end{dcases}&\cond{dck}{\dconv}
 \end{align}
\end{subequations}
while the specific heat and enthalpy\slash internal-energy are independent of the state parameter for both \conp~and \conv.
Following the now familiar collapsing of the specific-heat\slash concentration sum, and substituting in the energy source term gives:
\begin{subequations}
\begin{align}
 \frac{\partial\dot{T}}{\partial{V}} &= \frac{1}{\sum_{k=1}^{\ns} [C]_{k} {C_p}_{k}} \Biggl[ - \sum_{k=1}^{-1 + \ns} \left(H_{k} - \frac{W_{k} H_{\ns}}{W_{\ns}}\right) \frac{\partial \dot{\omega} }{\partial V }_{k} + & \nonumber \\
				     & \mathindent[largerindentand] \frac{1}{V} \frac{\text{d} T }{\text{d} t } \sum_{k=1}^{-1 + \ns} \left(- {C_p}_{\ns} + {C_p}_{k}\right) [C]_{k}\Biggr] & \text{\dconp} \\
 \frac{\partial\dot{T}}{\partial{P}} &= \frac{1}{\sum_{k=1}^{\ns} [C]_{k} {C_v}_{k}} \Biggl[- \sum_{k=1}^{-1 + \ns} \left(U_{k} - \frac{W_{k} U_{\ns}}{W_{\ns}}\right) \frac{\partial \dot{\omega} }{\partial P }_{k} - & \nonumber \\
				     & \mathindent[largerindentand] \frac{{C_v}_{\ns}}{T \Ru} \frac{\text{d} T }{\text{d} t } \Biggr] & \text{\dconv}
\end{align}
\end{subequations}
Finally, the species production rate derivative with respect to the state parameter will be computed in~\cref{s:d_spec_prod}.

\subsection{Molar source term derivatives}
\label{s:molar_source_derivatives}
\subsubsection{Temperature derivative}

The derivative of the net molar rate of production of a species $k$ with respect to temperature is:
\begin{align}
 \mathcal{J}_{k + 2, 1} &= \frac{\partial }{\partial T } \frac{\text{d} n_{k}}{\text{d} t} = \frac{\partial \dot{n}_k }{\partial T } \nonumber \\
 \intertext{or:}
   \mathcal{J}_{k + 2, 1} & = V \frac{\partial \dot{\omega}_{k} }{\partial T }
\end{align}

\subsubsection{Molar derivative}

Similarly, the derivative of the net molar rate of production of a species $k$ with respect to the moles of species $j$ is:
\begin{align}
 \mathcal{J}_{k + 2, j + 2} &= \frac{\partial }{\partial n_j } \frac{\text{d} n_{k}}{\text{d} t} = \frac{\partial \dot{n}_k }{\partial n_j } \nonumber \\
  \intertext{or:}
 \mathcal{J}_{k + 2, j + 2} & = V \frac{\partial \dot{\omega}_{k} }{\partial {n_j} }
\end{align}

\subsubsection{State parameter derivatives}

Finally, the derivative of the net molar rate of production of a species $k$ with respect to the state parameter is:
\begin{align}
  \mathcal{J}_{k + 2, 2} &= \frac{\partial }{\partial V } \frac{\text{d} n_{k}}{\text{d} t} = \frac{\partial \dot{n}_k }{\partial V } & \nonumber \\
			 &= V \frac{\partial \dot{\omega}_{k} }{\partial V } + \dot{\omega}_{k} & \text{\dconp}
  \intertext{and:}
  \mathcal{J}_{k + 2, 2} &= \frac{\partial }{\partial P } \frac{\text{d} n_{k}}{\text{d} t} = \frac{\partial \dot{n}_k }{\partial P } & \nonumber \\
			 &=V \frac{\partial \dot{\omega}_{k} }{\partial P } & \text{\dconv}
\end{align}

\subsection{State parameter source term derivatives}
\label{s:parameter_source_derivatives}
\subsubsection{Temperature derivative}

The derivatives of the thermodynamic state parameter source terms with respect to temperature are given by differentiating~\cref{e:param_complete}:
\begin{align}
 \mathcal{J}_{2, 1} &= \frac{\partial }{\partial T } \frac{\text{d} V}{\text{d} t} = \frac{\partial \dot{V} }{\partial T} & \nonumber \\
		    &= \frac{V}{[C]} \sum_{k=1}^{-1 + \ns} \left(1 - \frac{W_{k}}{W_{\ns}}\right) \left(\frac{\partial \dot{\omega} }{\partial T }_{k} + \frac{\dot{\omega}_{k}}{T}\right) \nonumber \\
		    &  \mathindent + \frac{V}{T} \left(\frac{\text{d} \dot{T} }{\text{d} T } - \frac{1}{T} \frac{\text{d} T }{\text{d} t }\right) & \text{\dconp}
\intertext{and:}
 \mathcal{J}_{2, 1} &= \frac{\partial }{\partial T } \frac{\text{d} V}{\text{d} t} = \frac{\partial \dot{P} }{\partial T} & \nonumber \\
		    &= \Ru \sum_{k=1}^{-1 + \ns} \left(1 - \frac{W_{k}}{W_{\ns}}\right) \left(T \frac{\partial \dot{\omega} }{\partial T }_{k} + \dot{\omega}_{k}\right) & \nonumber \\
		    & \mathindent + \frac{P}{T} \left(\frac{\text{d} \dot{T} }{\text{d} T } - \frac{1}{T} \frac{\text{d} T }{\text{d} t }\right) & \text{\dconv}
\end{align}

\subsubsection{Molar derivative}
Similarly, the derivatives of the thermodynamic state parameter source terms with respect to the moles of species $j$ are:
\begin{align}
 \mathcal{J}_{2, j + 2} &= \frac{\partial }{\partial n_j } \frac{\text{d} V}{\text{d} t} = \frac{\partial \dot{V} }{\partial n_j} & \nonumber \\
		    &= V \left(\frac{1}{[C]} \sum_{k=1}^{-1 + \ns} \left(1 - \frac{W_{k}}{W_{\ns}}\right) \frac{\partial \dot{\omega} }{\partial {n_j} }_{k} + \frac{1}{T} \frac{\text{d} \dot{T} }{\text{d} {n_j} }\right) & \text{\dconp}
\intertext{and:}
 \mathcal{J}_{2, j + 2} &= \frac{\partial }{\partial n_j } \frac{\text{d} V}{\text{d} t} = \frac{\partial \dot{P} }{\partial n_j} & \nonumber \\
		    &=  \frac{P}{T} \frac{\text{d} \dot{T} }{\text{d} {n_j} } + T \Ru \sum_{k=1}^{-1 + \ns} \left(1 - \frac{W_{k}}{W_{\ns}}\right) \frac{\partial \dot{\omega} }{\partial {n_j} }_{k} & \text{\dconv}
\end{align}

\subsubsection{State parameter derivative}
Finally, the derivative of the the thermodynamic state parameter source terms with respect to themselves are:
\begin{align}
 \mathcal{J}_{2, 2} &= \frac{\partial }{\partial V } \frac{\text{d} V}{\text{d} t} = \frac{\partial \dot{V} }{\partial V} & \nonumber \\
		    &= \frac{1}{[C]} \sum_{k=1}^{-1 + \ns} \left(1 - \frac{W_{k}}{W_{\ns}}\right) \left(V \frac{\partial \dot{\omega} }{\partial V }_{k} + \dot{\omega}_{k}\right) \nonumber \\
		    & \mathindent + \frac{1}{T} \left(V \frac{\text{d} \dot{T} }{\text{d} V } + \dot{T}\right) & \text{\dconp} \\
 \intertext{and:}
 \mathcal{J}_{2, 2} &= \frac{\partial }{\partial P } \frac{\text{d} P}{\text{d} t} = \frac{\partial \dot{P} }{\partial P} & \nonumber \\
		    &= T \Ru \sum_{k=1}^{-1 + \ns} \left(1 - \frac{W_{k}}{W_{\ns}}\right) \frac{\partial \dot{\omega} }{\partial P }_{k} & \nonumber \\ 
		    & \mathindent + \frac{1}{T} \left(P \frac{\text{d} \dot{T} }{\text{d} P } + \dot{T}\right) & \text{\dconv}
\end{align}



\subsection{Species production rate derivatives}
\label{s:d_spec_prod}
First, we evaluate the derivatives of the net species production rate equation~\cref{e:spec_rop}, with respect to temperature:
\begin{align}
 \label{e:dwdot_dT}
 \lhs{dwdot}{\frac{\partial \dot{\omega} }{\partial T }_{k}} &= \rhs{dwdot}{\sum_{i=1}^{\nr} \left(\nu_{k,i} R_{i} \frac{\partial c }{\partial T }_{i} + \nu_{k,i} \frac{\partial R }{\partial T }_{i} c_{i}\right)} & \cond{dwdot}{,}
\end{align}
other species:
\begin{align}
 \label{e:dwdot_dnj}
 \lhs{dwdot}{\frac{\partial \dot{\omega} }{\partial {n_j} }_{k}} &= \rhs{dwdot}{\sum_{i=1}^{\nr} \left(\nu_{k,i} R_{i} \frac{\partial c }{\partial {n_j} }_{i} + \nu_{k,i} \frac{\partial R }{\partial {n_j} }_{i} c_{i}\right)} & \cond{dwdot}{,}
\end{align}
and the state parameter:
\begin{subequations}
 \label{e:dwdot_de}
 \begin{align}
  \lhs{dwdot}{\frac{\partial \dot{\omega} }{\partial V }_{k}} &= \rhs{dwdot}{\sum_{i=1}^{\nr} \left(\nu_{k,i} R_{i} \frac{\partial c }{\partial V }_{i} + \nu_{k,i} \frac{\partial R }{\partial V }_{i} c_{i}\right)} & \cond{dwdot}{\dconp}\\
  \lhs{dwdot}{\frac{\partial \dot{\omega} }{\partial P }_{k}} &= \rhs{dwdot}{\sum_{i=1}^{\nr} \left(\nu_{k,i} R_{i} \frac{\partial c }{\partial P }_{i} + \nu_{k,i} \frac{\partial R }{\partial P }_{i} c_{i}\right)} & \cond{dwdot}{\dconv}
 \end{align}
\end{subequations}

\subsection{Rate of progress derivatives}
\subsubsection{Temperature derivative}
\label{s:dri_dt}
Next, the rate of progress derivatives are evaluated, first with respect to temperature:
\begin{equation}
 \frac{\partial R }{\partial T }_{i} = \frac{\partial}{\partial T}\left({k_{f, i}} \prod_{k=1}^{\ns} [C]_{k}^{\nu^{\prime}_{k,i}}\right)
\end{equation}
The derivative of the forward reaction rate coefficient is:
\begin{equation}
 \label{e:dkf_dt}
 \frac{\text{d} {k_f} }{\text{d} T }_{i} = \frac{{k_{f, i}}}{T} \left(\beta_{i} + \frac{{E_{a}}_{i}}{T \Ru}\right)
\end{equation}
While evaluating derivatives for the Jacobian, it is important to expand terms---e.g., concentration, production rate, etc.---related to the last species to obtain the correct derivative, in this case using \cref{e:last_spec_conc}:
\begin{equation}
 {R_f} = \left(\left(- \sum_{k=1}^{\ns  - 1} [C]_{k} + \frac{P}{T \Ru}\right)^{\nu^{\prime}_{\ns,i}}\right) {k_{f, i}} \prod_{k=1}^{\ns  - 1} [C]_{k}^{\nu^{\prime}_{k,i}}
\end{equation}

Hence, the full derivative of the forward rate of progress is:
\begin{equation}
 \label{e:dropfdt_1}
 \frac{\partial {R_f} }{\partial T }_{i} = \frac{\text{d} {k_f} }{\text{d} T }_{i} \prod_{k=1}^{\ns} [C]_{k}^{\nu^{\prime}_{k,i}} - \frac{[C] \nu^{\prime}_{\ns,i}}{T} [C]_{\ns}^{\nu^{\prime}_{\ns,i} - 1} {k_{f, i}} \prod_{k=1}^{\ns  - 1} [C]_{k}^{\nu^{\prime}_{k,i}}
\end{equation}
By defining a temporary variable $S^{\prime}_{l}$ (for species $l = \left\{j, \ns\right\}$):
\begin{equation}
 \label{e:s_temp}
 S^{\prime}_{l} = \nu^{\prime}_{l,i} [C]_{l}^{\nu^{\prime}_{l,i} - 1} \prod_{\substack{1 \leq l \leq l - 1\\l + 1 \leq l \leq \ns}} [C]_{l}^{\nu^{\prime}_{l,i}}
\end{equation}
and using~\cref{e:dkf_dt}, \cref{e:dropfdt_1} can be simplified to:
\begin{equation}
 \label{e:dropf_dt}
 \frac{\partial {R_f} }{\partial T }_{i} = - \frac{[C] S^{\prime}_{\ns}}{T} {k_{f, i}} + \frac{{R_{f, i}}}{T} \left(\beta_{i} + \frac{{E_{a}}_{i}}{T \Ru}\right)
\end{equation}

Starting from~\cref{e:ropr} and applying the same expansion of the last species' concentration as previously, the temperature derivative of the reverse rate of progress is:
\begin{equation}
 \label{e:droprdt_1}
 \frac{\partial {R_r} }{\partial T }_{i} = \frac{\text{d} {k_r} }{\text{d} T }_{i} \prod_{k=1}^{\ns} [C]_{k}^{\nu^{\prime\prime}_{k,i}} - \frac{[C] \nu^{\prime\prime}_{\ns,i}}{T} [C]_{\ns}^{\nu^{\prime\prime}_{\ns,i} - 1} {k_{f, i}} \prod_{k=1}^{\ns  - 1} [C]_{k}^{\nu^{\prime\prime}_{k,i}}
\end{equation}
Next, \cref{e:kr} is considered to obtain the temperature derivative of the non-explicit reversible reaction rate coefficient:
\begin{align}
 \frac{\text{d} {k_r} }{\text{d} T }_{i} &= \left(- \frac{1}{{K_{c, i}}} \frac{\text{d} {K_{c, i}} }{\text{d} T } \frac{{k_f}_i}{{K_c}_i} + \frac{1}{{K_{c, i}}}\frac{\text{d} {k_{f, i}} }{\text{d} T } \right) \nonumber \\
				         &= \left(- \frac{1}{{K_{c, i}}} \frac{\text{d} {K_{c, i}} }{\text{d} T } + \frac{1}{T} \left(\beta_{i} + \frac{{E_{a}}_{i}}{T \Ru}\right)\right) {k_{r, i}}
\end{align}
The temperature derivative of the equilibrium constant is:
\begin{equation}
 \frac{\text{d} {K_{c, i}} }{\text{d} T } = {K_{c, i}} \sum_{k=1}^{\ns} \nu_{k,i} \frac{\text{d} B }{\text{d} T }_{k} \;,
\end{equation}
with:
\begin{equation}
 \frac{\text{d} B }{\text{d} T }_{k} = T \left(T \left(\frac{T a_{k,4}}{5} + \frac{a_{k,3}}{4}\right) + \frac{a_{k,2}}{3}\right) + \frac{a_{k,1}}{2} + \frac{1}{T} \left(a_{k,0} - 1 + \frac{a_{k,5}}{T}\right)
\end{equation}
giving:
\begin{equation}
\label{e:dkr_dt}
\frac{\text{d} {k_r} }{\text{d} T }_{i} = \left(- \sum_{k=1}^{\ns} \nu_{k,i} \frac{\text{d} B }{\text{d} T }_{k} + \frac{1}{T} \left(\beta_{i} + \frac{{E_{a}}_{i}}{T \Ru}\right)\right) {k_{r, i}} \;,
\end{equation}
and using an temporary value $S^{\prime\prime}_{\ns}$---defined analagously to \cref{e:s_temp} for the reverse direction---we obtain:
\begin{equation}
\label{e:dropr_dt}
\frac{\partial {R_r} }{\partial T }_{i} = \left(- \sum_{k=1}^{\ns} \nu_{k,i} \frac{\text{d} B }{\text{d} T }_{k} + \frac{1}{T} \left(\beta_{i} + \frac{{E_{a}}_{i}}{T \Ru}\right)\right) {R_{r, i}} - \frac{[C] S^{\prime\prime}_{\ns}}{T} {k_{r, i}}
\end{equation}

Finally, combining~\cref{e:dropf_dt,e:dropr_dt} gives the total temperature derivative of the net rate of progress:
\begin{align}
 \label{e:drop_dt}
 \frac{\partial R }{\partial T }_{i} =& \frac{{R_{f, i}}}{T} \left(\beta_{i} + \frac{{E_{a}}_{i}}{T \Ru}\right) - \left(- \sum_{k=1}^{\ns} \nu_{k,i} \frac{\text{d} B }{\text{d} T }_{k} + \frac{1}{T} \left(\beta_{i} + \frac{{E_{a}}_{i}}{T \Ru}\right)\right) {R_{r, i}} \nonumber \\
				      &\qquad + \frac{[C] S^{\prime\prime}_{\ns}}{T} {k_{r, i}} - \frac{[C] S^{\prime}_{\ns}}{T} {k_{f, i}}
\end{align}
\subsubsection{Molar derivative}
\label{s:dri_dnj}
The derivative of the forward rate of progress with respect to the amount of moles of a species $j$ is:
\begin{equation}
 \label{e:dropf_dnj_1}
 \frac{d}{d n_{k}} {R_{f_i}} = \left(\frac{\partial}{\partial n_{j}} \prod_{k=1}^{\ns} [C]_{k}^{\nu^{\prime}_{k,i}}\right) {k_{f, i}} \;.
\end{equation}

Combining~\cref{e:dropf_dnj_1,e:dck_dnj} the molar derivative of the forward rate of progress is found as:
\begin{equation}
 \frac{\partial {R_{f, i}} }{\partial {n_j} } = {k_{f, i}} \sum_{k=1}^{\ns} \left(- \frac{\delta_{\ns k}}{V} + \frac{\delta_{j k}}{V}\right) \nu^{\prime}_{k,i} [C]_{k}^{\nu^{\prime}_{k,i} - 1} \prod_{\substack{1 \leq l \leq k - 1\\k + 1 \leq l \leq \ns}} [C]_{l}^{\nu^{\prime}_{l,i}}
\end{equation}
with the same temporary variables $S^{\prime}_{j}$ and $S^{\prime}_{\ns}$ as defined previously, this can be simplified to:
\begin{equation}
 \label{e:dropf_dnj}
 \frac{\partial {R_{f, i}} }{\partial {n_j} } = \frac{{k_{f, i}}}{V} \left(- S^{\prime}_{\ns} + S^{\prime}_{j}\right)
\end{equation}
and by the same process, the reverse rate of progress derivative is found to be:
\begin{equation}
 \label{e:dropr_dnj}
 \frac{\partial {R_{r, i}} }{\partial {n_j} } = \frac{{k_{r, i}}}{V} \left(- S^{\prime\prime}_{\ns} + S^{\prime\prime}_{j}\right)
\end{equation}
and the net rate of progress molar derivative:
\begin{equation}
 \frac{\partial R_{i} }{\partial {n_j} } = - \frac{{k_{r, i}}}{V} \left(- S^{\prime\prime}_{\ns} + S^{\prime\prime}_{j}\right) + \frac{{k_{f, i}}}{V} \left(- S^{\prime}_{\ns} + S^{\prime}_{j}\right)
\end{equation}

\subsubsection{State parameter derivatives}
\label{s:dri_de}
Following the outline of~\cref{s:dri_dnj}---while utilising~\cref{e:dck_de} for the species concentration derivatives---the forward rate of progress derivatives are:
\begin{subequations}
 \begin{align}
  \lhs{dri_de}{\frac{\partial {R_{f, i}} }{\partial V }} &= \rhs{dri_de}{\frac{[C] S^{\prime}_{\ns}}{V} {k_{f, i}} - \frac{{R_{f, i}}}{V} \sum_{k=1}^{\ns} \nu^{\prime}_{k,i}}&\cond{dri_de}{\dconp}\\
  \lhs{dri_de}{\frac{\partial {R_{f, i}} }{\partial P }} &= \rhs{dri_de}{\frac{S^{\prime}_{\ns} {k_{f, i}}}{T \Ru}}&\cond{dri_de}{\dconv}
 \end{align}
\end{subequations}
and similarly, the reverse rate of progress derivatives:
\begin{subequations}
 \begin{align}
  \lhs{dri_de}{\frac{\partial {R_{r, i}} }{\partial V }} &= \rhs{dri_de}{\frac{[C] S^{\prime\prime}_{\ns}}{V} {k_{r, i}} - \frac{{R_{r, i}}}{V} \sum_{k=1}^{\ns} \nu^{\prime\prime}_{k,i}}&\cond{dri_de}{\dconp}\\
  \lhs{dri_de}{\frac{\partial {R_{r, i}} }{\partial P }} &= \rhs{dri_de}{\frac{S^{\prime\prime}_{\ns} {k_{r, i}}}{T \Ru}}&\cond{dri_de}{\dconv}
 \end{align}
\end{subequations}
giving finally:
\begin{subequations}
 \label{e:dropi_de}
 \begin{align}
  \lhs{dri_de}{\frac{\partial {R}_{i} }{\partial V }} &= \rhs{dri_de}{\frac{1}{V}\Biggl( [C]\left(S^{\prime}_{\ns} {k_{f, i}} - S^{\prime\prime}_{\ns} {k_{r, i}}\right) + } & \nonumber\\
  & \rhs{dri_de}{\mathindent[farth] \left({R_{r, i}} \sum_{k=1}^{\ns} \nu^{\prime\prime}_{k,i} - {R_{f, i}} \sum_{k=1}^{\ns} \nu^{\prime}_{k,i}\right) \Biggr)}&\cond{dri_de}{\dconp}\\
  \lhs{dri_de}{\frac{\partial {R}_{i} }{\partial P }} &= \rhs{dri_de}{\frac{1}{T \Ru} \left(S^{\prime}_{\ns} {k_{f, i}} - S^{\prime\prime}_{\ns} {k_{r, i}}\right)}&\cond{dri_de}{\dconv}
 \end{align}
\end{subequations}

\subsubsection{Pressure-dependent reactions}
\label{s:dri_pdep}
The P-Log and Chebyshev pressure-dependent reactions, described by~\cref{e:kf_plog,e:kf_cheb} require separate treatment from the derivatives of strictly Arrhenius-based reaction rate formulations examined in~\cref{s:dri_dt,s:dri_dnj,s:dri_de}.
Beginning with P-Log reactions, the derivative of the forward rate coefficient with respect to temperature is:
\begin{align}
 \frac{\partial {k_{f, i}} }{\partial T } &= \biggl( \frac{1}{k_{1}} \frac{\text{d} k_1 }{\text{d} T } + \frac{1}{- \log{\left (P_{1} \right )} + \log{\left (P_{2} \right )}} \left(- \frac{1}{k_{1}} \frac{\text{d} k_1 }{\text{d} T } + \frac{1}{k_{2}} \frac{\text{d} k_2 }{\text{d} T }\right) \times  \Bigl( \nonumber \\
					 & \mathindent - \log{\left (P_{1} \right )} + \log{\left (P \right )}\Bigr)\biggr) {k_{f, i}}
\end{align}
while the derivatives of $k_{1}$ and $k_{2}$ are given by~\cref{e:dkf_dt}---with the corresponding Arrhenius parameters (see~\cref{e:plog_k1,e:plog_k2}) substituted in.
Simplifying gives:
\begin{align}
 \label{e:dkf_plog_dt}
 \frac{\partial {k_{f, i}} }{\partial T } &= \frac{{k_{f, i}}}{T} \Biggl(\frac{\left(- \log{\left (P_{1} \right )} + \log{\left (P \right )}\right) \left(- \beta_1 + \beta_2 - \frac{E_{a_1}}{T \Ru} + \frac{E_{a_2}}{T \Ru}\right)}{- \log{\left (P_{1} \right )} + \log{\left (P_{2} \right )}} + \nonumber \\
					 & \mathindent[morespac] \beta_1 + \frac{E_{a_1}}{T \Ru}\Biggr)
\end{align}
Similar to~\cref{s:dri_dt},~\cref{e:dkf_plog_dt} may be substituted into~\cref{e:dropfdt_1,e:droprdt_1} to give:
\begin{align}
 \frac{\partial R_{i} }{\partial T } &= \left(\frac{\left(- \log{\left (P_{1} \right )} + \log{\left (P \right )}\right) \left(- \beta_1 + \beta_2 - \frac{E_{a_1}}{T \Ru} + \frac{E_{a_2}}{T \Ru}\right)}{- \log{\left (P_{1} \right )} + \log{\left (P_{2} \right )}} + \beta_1 + \frac{E_{a_1}}{T \Ru} \right) \times \nonumber \\
				     & \mathindent \left(\frac{{R_{f, i}} - {R_{r, i}}}{T}\right) + {R_{r, i}} \sum_{k=1}^{\ns} \nu_{k,i} \frac{\text{d} B }{\text{d} T }_{k} + \frac{[C]}{T} \left(S^{\prime\prime}_{\ns} {k_{r, i}} - S^{\prime}_{\ns} {k_{f, i}}\right)
\end{align}

For \conv~problems, the P-Log forward rate coefficient is also a function of the pressure:
\begin{equation}
 \frac{\partial {k_{f, i}} }{\partial P } = \frac{\left(- \log{\left (k_{1} \right )} + \log{\left (k_{2} \right )}\right) {k_{f, i}}}{P \left(- \log{\left (P_{1} \right )} + \log{\left (P_{2} \right )}\right)}
\end{equation}
substituting this into the net rate of progress derivative gives:
\begin{align}
 \frac{\partial R_{i} }{\partial P } &= \frac{1}{T \Ru} \left(- S^{\prime\prime}_{\ns} {k_{r, i}} + S^{\prime}_{\ns} {k_{f, i}}\right) + & \nonumber \\
				     &  \frac{\left(- \log{\left (k_{1} \right )} + \log{\left (k_{2} \right )}\right) \left({R_{f, i}} - {R_{r, i}}\right)}{P \left(- \log{\left (P_{1} \right )} + \log{\left (P_{2} \right )}\right)} & \text{\dconv}
\end{align}

Next, the temperature derivative of Chevbyshev reactions will be considered:
\begin{align}
 \frac{\partial {k_{f, i}} }{\partial T } &= \log{\left (10 \right )} {k_{f, i}} \sum_{\substack{1 \leq l \leq N_{P}\\1 \leq j \leq N_{T}}} \frac{\text{d} \tilde{T} }{\text{d} T } \left(j - 1\right) \mathcal{T}_{l - 1}\left(\tilde{P}\right) U_{j - 2}\left(\tilde{T}\right) \eta_{l,j}
\end{align}
where $U_n$ is the Chebyshev polynomial of the second kind of degree n, expressed as:
\begin{equation}
 U_n \left(x\right) = \frac{\sin\left(\left(n + 1\right) \arccos x\right)}{\sin \left(\arccos x \right)}
\end{equation}
and:
\begin{equation}
 \frac{\text{d} \tilde{T}}{\text{d} T} = \frac{-2}{T^2 \left(-\frac{1}{T_{\min}} + \frac{1}{T_{\text{max}}}\right)}
\end{equation}
giving:
\begin{equation}
 \frac{\partial {k_{f, i}} }{\partial T } = \log{\left (10 \right )} {k_{f, i}} \sum_{\substack{1 \leq l \leq N_{P}\\1 \leq j \leq N_{T}}} - \frac{2 \mathcal{T}_{l - 1}\left(\tilde{P}\right) U_{j - 2}\left(\tilde{T}\right) \eta_{l,j}}{T^{2} \left(- \frac{1}{T_{\min}} + \frac{1}{T_{\max}}\right)} \left(j - 1\right)
\end{equation}
Following the same process for Arrhenius-based reactions (see~\cref{s:dri_dt}), the derivative of the net rate of progress with respect to temperature is:
\begin{align}
 \frac{\partial R }{\partial T }_{i} &= \sum_{\substack{1 \leq l \leq N_{P}\\1 \leq j \leq N_{T}}} \left[ - \frac{2 \log{\left (10 \right )} \mathcal{T}_{l - 1}\left(\tilde{P}\right) U_{j - 2}\left(\tilde{T}\right) \eta_{l,j}}{T^{2} \left(- \frac{1}{T_{\min}} + \frac{1}{T_{\max}}\right)} \left(j - 1\right) \right] \left({R_{f, i}} - {R_{r, i}}\right) + \nonumber \\
				     & \mathindent \sum_{k=1}^{\ns} \nu_{k,i} \frac{\text{d} B }{\text{d} T }_{k} {R_{r, i}} + \frac{[C]}{T} \left(S^{\prime\prime}_{\ns} {k_{r, i}} - S^{\prime}_{\ns} {k_{f, i}}\right)
\end{align}

For \conv~problems, we again must consider the derivative of the forward rate coefficient with respect to pressure:
\begin{equation}
 \frac{\partial {k_{f, i}} }{\partial P } = \log{\left (10 \right )} {k_{f, i}} \sum_{\substack{1 \leq l \leq N_{P}\\1 \leq j \leq N_{T}}} \frac{\text{d} \tilde{P} }{\text{d} P } \left(l - 1\right) \mathcal{T}_{j - 1}\left(\tilde{T}\right) U_{l - 2}\left(\tilde{P}\right) \eta_{l,j}
\end{equation}
with:
\begin{equation}
 \frac{\text{d} \tilde{P} }{\text{d} P } = \frac{2}{P \left(\log{\left (P_{\max} \right )} - \log{\left (P_{\min} \right )}\right)}
\end{equation}
giving:
\begin{equation}
 \frac{\partial {k_{f, i}} }{\partial P } = \log{\left (10 \right )} {k_{f, i}} \sum_{\substack{1 \leq l \leq N_{P}\\1 \leq j \leq N_{T}}} \frac{2 \left(l - 1\right) \mathcal{T}_{j - 1}\left(\tilde{T}\right) U_{l - 2}\left(\tilde{P}\right) \eta_{l,j}}{P \left(\log{\left (P_{\max} \right )} - \log{\left (P_{\min} \right )}\right)}
\end{equation}
and:
\begin{align}
 \frac{\partial R_{i} }{\partial P } &= \log{\left (10 \right )} \left({R_{f, i}} - {R_{r, i}}\right) \sum_{\substack{1 \leq l \leq N_{P}\\1 \leq j \leq N_{T}}} \frac{2 \left(l - 1\right) \mathcal{T}_{j - 1}\left(\tilde{T}\right) U_{l - 2}\left(\tilde{P}\right) \eta_{l,j}}{P \left(\log{\left (P_{\max} \right )} - \log{\left (P_{\min} \right )}\right)} + \nonumber \\
				     & \mathindent \frac{1}{T \Ru} \left(- S^{\prime\prime}_{\ns} {k_{r, i}} + S^{\prime}_{\ns} {k_{f, i}}\right)
\end{align}



\subsection{Pressure modification\slash Falloff function derivatives}
\label{s:dpmod_fall}
\subsubsection{Elementary reactions}
For elementary reactions, all derivatives of the pressure-modification term are zero:
\begin{align}
\lhs{dci_elem}{\frac{\partial c_{i} }{\partial T }} &= \rhs{dci_elem}{0} & \cond{dci_elem}{,} \label{e:dci_elem_dt} \\
\lhs{dci_elem}{\frac{\partial c_{i} }{\partial {n_j} }} &= \rhs{dci_elem}{0} & \cond{dci_elem}{,} \label{e:dci_elem_dnj} \\
\end{align}
and:
\begin{subequations}
 \label{e:dci_elem_de}
 \begin{align}
  \lhs{dci_elem}{\frac{\partial c_{i} }{\partial V }} &= \rhs{dci_elem}{0} & \cond{dci_elem}{\dconp} \\
  \lhs{dci_elem}{\frac{\partial c_{i} }{\partial P }} &= \rhs{dci_elem}{0} & \cond{dci_elem}{\dconv}
 \end{align}
\end{subequations}


\subsubsection{Third-body enhanced reactions}
For a third-body enhanced reaction~\cref{e:thd_mix} is differentiated to give:
\begin{align}
 \lhs{dci_thd_mix}{\frac{\partial [X]_i }{\partial T }} &= \rhs{dci_thd_mix}{-\frac{[C] \alpha_{\ns,i}}{T}} & \cond{dci_thd_mix}{,} \label{e:dci_thd_dt} \\
 \lhs{dci_thd_mix}{\frac{\partial [X]_i }{\partial n_{j}}} &= \rhs{dci_thd_mix}{\frac{1}{V} \left(- \alpha_{\ns,i} + \alpha_{j,i}\right)} & \cond{dci_thd_mix}{,} \label{e:dci_thd_dnj}
\end{align}
and:
\begin{subequations}
 \label{e:dci_thd_de}
 \begin{align}
  \lhs{dci_thd_mix}{\frac{\partial [X]_i }{\partial V }} &= \rhs{dci_thd_mix}{\frac{1}{V} \left([C] \alpha_{\ns,i} - [X]_{i}\right)} & \cond{dci_thd_mix}{\dconp}\\
  \lhs{dci_thd_mix}{\frac{\partial [X]_i }{\partial P }} &= \rhs{dci_thd_mix}{\frac{\alpha_{\ns,i}}{T \Ru}} & \cond{dci_thd_mix}{\dconv}
 \end{align}
\end{subequations}

If $\alpha_{j,i} = 1$ for all species $j$,~\cref{e:dci_thd_dt,e:dci_thd_dnj,e:dci_thd_de} may be simplified to:
\begin{align}
 \lhs{dci_thd_unity}{\frac{\partial c_{i} }{\partial T }} &= \rhs{dci_thd_unity}{- \frac{[C]}{T}}& \cond{dci_thd_unity}{,} \label{e:dci_thd_unity_dt} \\
 \lhs{dci_thd_unity}{\frac{\partial c_{i} }{\partial n_{j} }} &= \rhs{dci_thd_unity}{0}[c] & \cond{dci_thd_unity}{,}
\end{align}
and:
\begin{subequations}
 \label{e:dci_thd_unity_de}
 \begin{align}
  \lhs{dci_thd_unity}{\frac{\partial c_{i} }{\partial V }} &= \rhs{dci_thd_unity}{0}[c] & \cond{dci_thd_unity}{\dconp} \\
  \lhs{dci_thd_unity}{\frac{\partial c_{i} }{\partial P }} &= \rhs{dci_thd_unity}{\frac{1}{T \Ru}}& \cond{dci_thd_unity}{\dconv}
 \end{align}
\end{subequations}

If the third-body is the $m$th species,~\cref{e:dci_thd_dt,e:dci_thd_dnj,e:dci_thd_de} become:
\begin{align}
 \lhs{dci_thd_spec}{\frac{\partial c_{i} }{\partial T }} &= \rhs{dci_thd_spec}{-\frac{\delta_{\ns m}}{T} [C]}&\cond{dci_thd_spec}{,} \label{e:dci_thd_spec_dt} \\
 \lhs{dci_thd_spec}{\frac{\partial c_{i} }{\partial n_j }} &= \rhs{dci_thd_spec}{\frac{1}{V} \left(- \delta_{\ns m} + \delta_{j m}\right)}&\cond{dci_thd_spec}{,} \label{e:dci_thd_spec_dnj} \\
\end{align}
and:
\begin{subequations}
\label{e:dci_thd_spec_de}
\begin{align}
\lhs{dci_thd_spec}{\frac{\partial c_{i} }{\partial V }} &= \rhs{dci_thd_spec}{\frac{\delta_{\ns m}}{V} \left([C] - [C]_{\ns}\right) + \frac{[C]_{m}}{V} \left(\delta_{\ns m} - 1\right)} & \cond{dci_thd_spec}{\dconp} \\
\lhs{dci_thd_spec}{\frac{\partial c_{i} }{\partial P }} &= \rhs{dci_thd_spec}{\frac{\delta_{\ns m}}{T \Ru}} & \cond{dci_thd_spec}{\dconv}
\end{align}
\end{subequations}

\subsubsection{Unimolecular\slash~recombination fall-off reactions}
\label{s:dfall}
Derivatives of the third-body pressure modification term for unimolecular\slash~recombination fall-off reactions---given by~\cref{e:rxn_pressure}---are quite involved, and will be broken up over the next sections.
The derivative of full third-body pressure modification term for fall-off reactions will be given in this section, while~\cref{s:dchem} will provide the same derivatives for chemically-activated bimolecular reactions.
Subsequently, the derivatives of the reduced pressure and falloff blending factors will be explored in~\cref{s:dpr,s:dfi} respectively.

The derivatives of the unimolecular\slash~recombination fall-off reaction third-body pressure modification term are:
\begin{align}
 \label{e:dci_fall_dt}
 \lhs{dci_fall}{\frac{\partial c_{i} }{\partial T }} &= \rhs{dci_fall}{\frac{1}{P_{r, i} + 1} \left(P_{r, i} \frac{\partial F_{i} }{\partial T } + \frac{\partial P_{r, i} }{\partial T } \left(F_{i} - c_{i}\right)\right)} & \cond{dci_fall}{,}\\
 \label{e:dci_fall_dnj}
 \lhs{dci_fall}{\frac{\partial c_{i} }{\partial n_j}} &= \rhs{dci_fall}{\frac{1}{P_{r, i} + 1} \left(P_{r, i} \frac{\partial F_{i} }{\partial n[j] } + \frac{\partial P_{r, i} }{\partial n_j } \left(F_{i} - c_{i}\right)\right)} & \cond{dci_fall}{,}
\end{align}
and with respect to the state parameter:
\begin{subequations}
 \label{e:dci_fall_de}
 \begin{align}
  \lhs{dci_fall}{\frac{\partial c_i }{\partial V }} &= \rhs{dci_fall}{\frac{1}{P_{r, i} + 1} \left(P_{r, i} \frac{\partial F_{i} }{\partial V } + \frac{\partial P_{r, i} }{\partial V } \left(F_{i} - c_{i}\right)\right)} & \cond{dci_fall}{\dconp} \\
  \lhs{dci_fall}{\frac{\partial c_i }{\partial P }} &= \rhs{dci_fall}{\frac{1}{P_{r, i} + 1} \left(P_{r, i} \frac{\partial F_{i} }{\partial P } + \frac{\partial P_{r, i} }{\partial P } \left(F_{i} - c_{i}\right)\right)} & \cond{dci_fall}{\dconv}
 \end{align}
\end{subequations}

\subsubsection{Chemically-activated bimolecular reactions}
\label{s:dchem}

Similarly, the chemically-activated bimolecular reactions are as follows:
\begin{align}
 \label{e:dci_chem_dt}
 \lhs{dci_chem}{\frac{\partial c_i }{\partial T }} &= \rhs{dci_chem}{\frac{1}{P_{r, i} + 1} \left(\frac{\partial F_{i} }{\partial T } - \frac{\partial P_{r, i} }{\partial T } c_{i}\right)} & \cond{dci_chem}{,} \\
 \label{e:dci_chem_dnj}
 \lhs{dci_chem}{\frac{\partial c_i }{\partial n_j }} &= \rhs{dci_chem}{\frac{1}{P_{r, i} + 1} \left(\frac{\partial F_{i} }{\partial n_j } - \frac{\partial P_{r, i} }{\partial n_j } c_{i}\right)} & \cond{dci_chem}{,}
\end{align}
and:
\begin{subequations}
 \label{e:dci_chem_de}
 \begin{align}
  \lhs{dci_chem}{\frac{\partial c_i }{\partial V }} &= \rhs{dci_chem}{\frac{1}{P_{r, i} + 1} \left(\frac{\partial F_{i} }{\partial V } - \frac{\partial P_{r, i} }{\partial V } c_{i}\right)} & \cond{dci_chem}{\dconp} \\
  \lhs{dci_chem}{\frac{\partial c_i }{\partial P }} &= \rhs{dci_chem}{\frac{1}{P_{r, i} + 1} \left(\frac{\partial F_{i} }{\partial P } - \frac{\partial P_{r, i} }{\partial P } c_{i}\right)} & \cond{dci_chem}{\dconv}
 \end{align}
\end{subequations}

\subsubsection{Reduced pressure}
\label{s:dpr}

Next the reduced pressure derivatives in~\crefrange{e:dci_fall_dt}{e:dci_chem_de} will be evaluated; these depend on the form of the third-body efficiency used in~\cref{e:pr}, thus we will first work out the derivatives for each third-body efficiency form, and then develop a generalized form for compact representation.
For a third-body concentration based upon the entire mixture, (i.e.,~\cref{e:thd_mix}) the derivatives are:
\begin{align}
 \lhs{dpr_mix}{\frac{\partial P_{r, i} }{\partial T }} &= \rhs{dpr_mix}{\frac{P_{r, i}}{T} \left(\beta_{0} - \beta_{\infty} + \frac{E_{a, 0}}{T \Ru} - \frac{E_{a, \infty}}{T \Ru}\right) - \frac{[C] k_{0, i} \alpha_{\ns,i}}{T k_{\infty, i}}} & {,} \label{e:dpr_mix_dt} \\
 \lhs{dpr_mix}{\frac{\partial P_{r, i} }{\partial n_j }} &= \rhs{dpr_mix}{\frac{k_{0, i} \left(- \alpha_{\ns,i} + \alpha_{j,i}\right)}{k_{\infty, i} V}} & \cond{dpr_mix}{,} \label{e:dpr_mix_dnj}
\end{align}
and:
\begin{subequations}
 \label{e:dpr_mix_de}
 \begin{align}
  \lhs{dpr_mix}{\frac{\partial P_{r, i} }{\partial V }} &= \rhs{dpr_mix}{- \frac{P_{r, i}}{V} + \frac{[C] k_{0, i} \alpha_{\ns,i}}{V k_{\infty, i}}} & \cond{dpr_mix}{\dconp} \\
  \lhs{dpr_mix}{\frac{\partial P_{r, i} }{\partial P }} &= \rhs{dpr_mix}{\frac{k_{0, i} \alpha_{\ns,i}}{T k_{\infty, i} \Ru}} & \cond{dpr_mix}{\dconv}
 \end{align}
\end{subequations}

For a unity third-body concentration (i.e., all $\alpha_{k,i}=1$):
\begin{align}
 \lhs{dpr_unity}{\frac{\partial P_{r, i} }{\partial T }} &= \rhs{dpr_unity}{\frac{P_{r, i}}{T} \left(\beta_{0} - \beta_{\infty} - 1 + \frac{E_{a, 0}}{T \Ru} - \frac{E_{a, \infty}}{T \Ru}\right)} & \cond{dpr_unity}{,} \label{e:dpr_unity_dt} \\
 \lhs{dpr_unity}{\frac{\partial P_{r, i} }{\partial n_j }} &= \rhs{dpr_unity}{0} & \cond{dpr_unity}{,} \label{e:dpr_unity_dnj}
\end{align}
and:
\begin{subequations}
 \label{e:dpr_unity_de}
 \begin{align}
  \lhs{dpr_unity}{\frac{\partial P_{r, i} }{\partial V }} &= \rhs{dpr_unity}{0} & \cond{dpr_unity}{\dconp}\\
  \lhs{dpr_unity}{\frac{\partial P_{r, i} }{\partial P }} &= \rhs{dpr_unity}{\frac{k_{0, i}}{k_{\infty, i}} \frac{1}{\Ru T}} & \cond{dpr_unity}{\dconv}
 \end{align}
\end{subequations}

Finally, for species-based third-body (i.e., the species $m$ alone acts as the third body):
\begin{align}
 \lhs{dpr_spec}{\frac{\partial P_{r, i} }{\partial T }} &=\rhs{dpr_spec}{\frac{P_{r, i}}{T} \left(\beta_{0} - \beta_{\infty} + \frac{E_{a, 0}}{T \Ru} - \frac{E_{a, \infty}}{T \Ru}\right) - \frac{[C] k_{0, i} \delta_{\ns m}}{T k_{\infty, i}}} & \cond{dpr_spec}{,} \label{e:dpr_spec_dt} \\
 \lhs{dpr_spec}{\frac{\partial P_{r, i} }{\partial n_j }} &=\rhs{dpr_spec}{\frac{k_{0, i}}{V k_{\infty, i}} \left(- \delta_{\ns m} + \delta_{j m}\right)} & \cond{dpr_spec}{,} \label{e:dpr_spec_dnj}
\end{align}
and:
\begin{subequations}
 \label{e:dpr_spec_de}
 \begin{align}
  \lhs{dpr_spec}{\frac{\partial P_{r, i} }{\partial V }} &= \rhs{dpr_spec}{- \frac{P_{r, i}}{V} + \frac{[C] k_{0, i} \delta_{\ns m}}{V k_{\infty, i}}} & \cond{dpr_spec}{\dconp}\\
  \lhs{dpr_spec}{\frac{\partial P_{r, i} }{\partial P }} &= \rhs{dpr_spec}{\frac{k_{0, i} \delta_{\ns m}}{T k_{\infty, i} \Ru}} & \cond{dpr_spec}{\dconv}
 \end{align}
\end{subequations}

\Crefrange{e:dpr_mix_dt}{e:dpr_spec_de} may be recast in a more generalized form using temporary variables $\Theta_{P_{r, i}, \partial \ldots}$ and $\bar{\theta}_{P_{r, i}, \partial \ldots}$---where, for example, $\Theta_{P_{r,i}, \partial T}$ corresponds to part of the reduced-pressure derivative with respect to temperature---to replace terms in each individual form of the reduced-pressure derivatives:
\begin{align}
 \lhs{dpr_gen}{\frac{\partial P_{r, i} }{\partial T } } &= \rhs{dpr_gen}{P_{r, i} \Theta_{P_{r,i}, \partial T} + \bar{\theta}_{P_{r, i}, \partial T}} & \cond{dpr_gen}{,} \label{e:dpr_dt} \\
 \lhs{dpr_gen}{\frac{\partial P_{r, i} }{\partial n_j}} &= \rhs{dpr_gen}{\frac{k_{0, i}\bar{\theta}_{P_{r, i}, \partial n_j}}{V k_{\infty, i}}} & \cond{dpr_gen}{,} \label{e:dpr_dnj}
\end{align}
and:
\begin{subequations}
 \label{e:dpr_de}
 \begin{align}
  \lhs{dpr_gen}{\frac{\partial P_{r, i} }{\partial V}} &= \rhs{dpr_gen}{P_{r, i} \Theta_{P_{r,i}, \partial V} + \bar{\theta}_{P_{r, i}, \partial V}} & \cond{dpr_gen}{\dconp} \\
  \lhs{dpr_gen}{\frac{\partial P_{r, i} }{\partial P}} &= \rhs{dpr_gen}{\bar{\theta}_{P_{r, i}, \partial P}} & \cond{dpr_gen}{\dconv}
 \end{align}
\end{subequations}
with:
\begin{align}
 \label{e:dpr_theta_dt}
 \Theta_{P_{r,i}, \partial T} &=
 \begin{dcases}
  \frac{1}{T} \left(\beta_{0} - \beta_{\infty} + \frac{E_{a, 0}}{T \Ru} - \frac{E_{a, \infty}}{T \Ru}\right) & \text{if mix,} \\
  \frac{1}{T} \left(\beta_{0} - \beta_{\infty} - 1 + \frac{E_{a, 0}}{T \Ru} - \frac{E_{a, \infty}}{T \Ru}\right) & \text{if unity,} \\
  \frac{1}{T} \left(\beta_{0} - \beta_{\infty} + \frac{E_{a, 0}}{T \Ru} - \frac{E_{a, \infty}}{T \Ru}\right) & \text{if species,}
 \end{dcases} \\
 \label{e:dpr_theta_bar_dt}
 \bar{\theta}_{P_{r, i}, \partial T} &=
 \begin{dcases}
  \rhs{theta_pr_bar_dt}{- \frac{[C] k_{0, i} \alpha_{\ns,i}}{T k_{\infty, i}}} & \text{if mix,} \\
  \rhs{theta_pr_bar_dt}{0}[c] & \text{if unity,} \\
  \rhs{theta_pr_bar_dt}{- \frac{[C] k_{0, i} \delta_{\ns m}}{T k_{\infty, i}}} & \text{if species,}
 \end{dcases} \\
 \label{e:dpr_theta_bar_dnj}
 \bar{\theta}_{P_{r, i}, \partial n_j} &=
 \begin{dcases}
  \rhs{theta_pr_bar_dnj}{- \alpha_{\ns,i} + \alpha_{j,i}} & \text{if mix,} \\
  \rhs{theta_pr_bar_dnj}{0}[c] & \text{if unity,}  \\
  \rhs{theta_pr_bar_dnj}{- \delta_{\ns m} + \delta_{j m}} & \text{if species,}
 \end{dcases} \\
 \label{e:dpr_theta_dv}
 \Theta_{P_{r,i}, \partial V} &=
 \begin{dcases}
  \rhs{Theta_pr_bar_dv}{- \frac{1}{V}} & \text{if mix,} \\
  \rhs{Theta_pr_bar_dv}{0}[c] & \text{if unity,}  \\
  \rhs{Theta_pr_bar_dv}{- \frac{1}{V}} & \text{if species,}
 \end{dcases} \\
 \label{e:dpr_theta_bar_dv}
 \bar{\theta}_{P_{r, i}, \partial V} &=
 \begin{dcases}
  \rhs{theta_pr_bar_dv}{\frac{[C] k_{0, i} \alpha_{\ns,i}}{V k_{\infty, i}}} & \text{if mix,} \\
  \rhs{theta_pr_bar_dv}{0}[c] & \text{if unity,}  \\
  \rhs{theta_pr_bar_dv}{\frac{[C] k_{0, i} \delta_{\ns m}}{V k_{\infty, i}}} & \text{if species,}
 \end{dcases} \\
 \label{e:dpr_theta_dp}
 \bar{\theta}_{P_{r, i}, \partial P} &=
 \begin{dcases}
  \rhs{theta_pr_bar_dp}{\frac{k_{0, i} \alpha_{\ns,i}}{T k_{\infty, i} \Ru}} & \text{if mix,} \\
  \rhs{theta_pr_bar_dp}{\frac{k_{0, i}}{T k_{\infty, i} \Ru}} & \text{if unity,}  \\
  \rhs{theta_pr_bar_dp}{\frac{k_{0, i} \delta_{\ns m}}{T k_{\infty, i} \Ru}} & \text{if species,}
 \end{dcases}
\end{align}


\subsubsection{Falloff blending factor}
\label{s:dfi}

The falloff blending factor for all but Lindemann falloff reactions relies upon the reduced pressure derivatives developed in~\cref{s:dpr}.
First, the blending factor for each falloff reaction type will be differentiated along with any subcomponents (e.g., the $F_{cent}$ term for Troe falloff reactions).
Next, the reduced pressured derivatives given by~\cref{e:dpr_dt,e:dpr_dnj,e:dpr_de} will be substituted in, before finally deriving a generalized form for the falloff blending factor derivatives.

For Lindemann falloff reactions:
\begin{align}
 \lhs{dfi_lind}{\frac{\partial F_{i} }{\partial T }} &= \rhs{dfi_lind}{0} & \cond{dfi_lind}{,} \label{e:dfi_lind_dt} \\
 \lhs{dfi_lind}{\frac{\partial F_{i} }{\partial n_j }} &= \rhs{dfi_lind}{0} & \cond{dfi_lind}{,} \label{e:dfi_lind_dnj}
\end{align}
and:
\begin{subequations}
 \label{e:dfi_lind_de}
 \begin{align}
  \lhs{dfi_lind}{\frac{\partial F_{i} }{\partial V }} &= \rhs{dfi_lind}{0} & \cond{dfi_lind}{\dconp} \\
  \lhs{dfi_lind}{\frac{\partial F_{i} }{\partial P }} &= \rhs{dfi_lind}{0} & \cond{dfi_lind}{\dconv}
 \end{align}
\end{subequations}

For Troe falloff reactions:
\begin{align}
 \lhs{dfi_troe_1}{\frac{\partial F_{i} }{\partial T }} &= \rhs{dfi_troe_1}{\frac{\partial F_{i} }{\partial F_{cent} } \frac{\text{d} F_{cent} }{\text{d} T } + \frac{\partial F_{i} }{\partial P_{r, i} } \frac{\partial P_{r, i} }{\partial T }} & \cond{dfi_troe_1}{,} \label{e:dfi_troe_dt} \\
 \lhs{dfi_troe_1}{\frac{\partial F_{i} }{\partial n_j}} &= \rhs{dfi_troe_1}{\frac{\partial F_{i} }{\partial P_{r, i} } \frac{\partial P_{r, i} }{\partial n_j }} & \cond{dfi_troe_1}{,} \label{e:dfi_troe_dnj}
\end{align}
and:
\begin{subequations}
 \label{e:dfi_troe_de}
 \begin{align}
 \lhs{dfi_troe_1}{\frac{\partial F_{i} }{\partial V }} &= \rhs{dfi_troe_1}{\frac{\partial F_{i} }{\partial P_{r, i} } \frac{\partial P_{r, i} }{\partial V }} & \cond{dfi_troe_1}{\dconp} \\
 \lhs{dfi_troe_1}{\frac{\partial F_{i} }{\partial P }} &= \rhs{dfi_troe_1}{\frac{\partial F_{i} }{\partial P_{r, i} } \frac{\partial P_{r, i} }{\partial P }} & \cond{dfi_troe_1}{\dconv}
 \end{align}
\end{subequations}
where:
\begin{align}
 \frac{\partial F_{i} }{\partial F_{cent} } &= \frac{F_{i}}{\frac{A_{Troe}^{2}}{B_{Troe}^{2}} + 1} \left(\frac{2 A_{Troe} \log{\left (F_{cent} \right )}}{B_{Troe}^{2} \left(\frac{A_{Troe}^{2}}{B_{Troe}^{2}} + 1\right)} \left(\frac{A_{Troe}}{B_{Troe}} \frac{\partial B_{Troe} }{\partial F_{cent} } - \right. \right. & \nonumber \\
 & \mathindent[morespaceplease] \left. \frac{\partial A_{Troe} }{\partial F_{cent} }\right) + \frac{1}{F_{cent}} \left. \vphantom{\frac{F_{i}}{\frac{A_{Troe}^{2}}{B_{Troe}^{2}} + 1}} \right) & \text{,} \label{e:dfi_troe_dfcent_partial} \\
\frac{\text{d} F_{cent} }{\text{d} T } &= \frac{a}{T^{*}} \operatorname{exp}\left({- \frac{T}{T^{*}}}\right) - \frac{\operatorname{exp}\left({- \frac{T}{T^{***}}}\right)}{T^{***}} \left(- a + 1\right) + & \nonumber \\
 & \mathindent \frac{T^{**}}{T^{2}} \operatorname{exp}\left({- \frac{T^{**}}{T}}\right) & \text{,} \label{e:dfcent_dt} \\
\intertext{and:}
\frac{\partial F_{i} }{\partial P_{r, i} } &= \frac{2 F_{i} A_{Troe} \log{\left (F_{cent} \right )}}{B_{Troe}^{2} \left(\frac{A_{Troe}^{2}}{B_{Troe}^{2}} + 1\right)^{2}} \left(\frac{A_{Troe}}{B_{Troe}} \frac{\partial B_{Troe} }{\partial P_{r, i} } - \frac{\partial A_{Troe} }{\partial P_{r, i} }\right) \label{e:dfi_troe_dpr_partial}
\end{align}
where the derivatives of $A_{Troe}$ and $B_{Troe}$ are:
\begin{align}
 \frac{\partial A_{Troe} }{\partial F_{cent} } &= - \frac{0.67}{F_{cent} \log{\left (10 \right )}} & \text{,} \label{e:datroe_dfcent} \\
\frac{\partial B_{Troe} }{\partial F_{cent} } &= - \frac{1.1762}{F_{cent} \log{\left (10 \right )}} & \text{,} \label{e:dbtroe_dfcent} \\
\frac{\partial A_{Troe} }{\partial P_{r, i} } &= \frac{1}{P_{r, i} \log{\left (10 \right )}} & \text{,} \label{e:datroe_dpr} \\
\intertext{and:}
\frac{\partial B_{Troe} }{\partial P_{r, i} } &= - \frac{0.14}{P_{r, i} \log{\left (10 \right )}} & \text{.} \label{e:dbtroe_dpr} \\
\end{align}

Combining~\cref{e:datroe_dfcent,e:dbtroe_dfcent,e:datroe_dpr,e:dbtroe_dpr} with~\cref{e:dfi_troe_dfcent_partial,e:dfi_troe_dpr_partial} yields:
\begin{align}
\frac{\partial F_{i} }{\partial F_{cent} } &= - \frac{F_{i} B_{Troe}}{F_{cent} \left(A_{Troe}^{2} + B_{Troe}^{2}\right)^{2} \log{\left (10 \right )}} \left(2 A_{Troe} \left( 1.1762 A_{Troe} - \right. \right . & \nonumber \\
& \mathindent[s] \left. \left. 0.67 B_{Troe}\right) \log{\left (F_{cent} \right )} - B_{Troe} \left(A_{Troe}^{2} + B_{Troe}^{2}\right) \log{\left (10 \right )}\right) & \text{,} \label{e:dfi_troe_dfcent} \\
\frac{\partial F_{i} }{\partial P_{r, i} } &= - \frac{2 F_{i} A_{Troe} \left(\frac{0.14 A_{Troe}}{B_{Troe}} + 1\right) \log{\left (F_{cent} \right )}}{B_{Troe}^{2} P_{r, i} \left(\frac{A_{Troe}^{2}}{B_{Troe}^{2}} + 1\right)^{2} \log{\left (10 \right )}} \label{e:dfi_troe_dpr} &
\end{align}
\Cref{e:dfcent_dt,e:dfi_troe_dfcent,e:dfi_troe_dpr} can then be substituted into~\cref{e:dfi_troe_dt,e:dfi_troe_dnj,e:dfi_troe_de} to obtain a final form for the Troe-falloff blending function derivatives, which will be given as a generalized form at the end of this section.

For SRI falloff reactions:
\begin{align}
 \lhs{dfi_sri}{\frac{\partial F_{i} }{\partial T }} &= F_{i} \vast(\frac{X \left(- \frac{\operatorname{exp}\left({- \frac{T}{c}}\right)}{c} + \frac{a b}{T^{2}} \operatorname{exp}\left({- \frac{b}{T}}\right)\right)}{a \operatorname{exp}\left({- \frac{b}{T}}\right) + \operatorname{exp}\left({- \frac{T}{c}}\right)} + \nonumber \\
 & \mathindent \frac{\partial P_{r, i} }{\partial T } \frac{\text{d} X }{\text{d} P_{r, i} } \log{\left (a \operatorname{exp}\left({- \frac{b}{T}}\right) + \operatorname{exp}\left({- \frac{T}{c}}\right) \right )}+ \frac{e}{T} \vast) & , \label{e:dfi_sri_dt} \\
 \lhs{dfi_sri}{\frac{\partial F_{i} }{\partial {n_j} }} &= \rhs{dfi_sri}{F_{i} \frac{\partial P_{r, i} }{\partial {n_j} } \frac{\text{d} X }{\text{d} P_{r, i} } \log{\left (a \operatorname{exp}\left({- \frac{b}{T}}\right) + \operatorname{exp}\left({- \frac{T}{c}}\right) \right )}} & , \label{e:dfi_sri_dnj}
\end{align}
and:
\begin{subequations}
 \label{e:dfi_sri_de}
 \begin{align}
 \lhs{dfi_sri}{\frac{\partial F_{i} }{\partial V }} &= \rhs{dfi_sri}{F_{i} \frac{\partial P_{r, i} }{\partial V } \frac{\text{d} X }{\text{d} P_{r, i} } \log{\left (a \operatorname{exp}\left({- \frac{b}{T}}\right) + \operatorname{exp}\left({- \frac{T}{c}}\right) \right )}} & \cond{dfi_sri}{\dconp} \\
 \lhs{dfi_sri}{\frac{\partial F_{i} }{\partial P }} &= \rhs{dfi_sri}{F_{i} \frac{\partial P_{r, i} }{\partial P } \frac{\text{d} X }{\text{d} P_{r, i} } \log{\left (a \operatorname{exp}\left({- \frac{b}{T}}\right) + \operatorname{exp}\left({- \frac{T}{c}}\right) \right )}} & \cond{dfi_sri}{\dconv}
 \end{align}
\end{subequations}
where:
\begin{align}
 \frac{\text{d} X }{\text{d} P_{r, i} } &= - \frac{2 X^{2} \log{\left (P_{r, i} \right )}}{P_{r, i} \log^{2}{\left (10 \right )}} &, \\
 \frac{\partial X}{\partial n_j} &= \frac{\partial P_{r, i} }{\partial {n_j} } \frac{\text{d} X }{\text{d} P_{r, i} }
\end{align}

As with the reduced-pressure derivatives,~\cref{e:dfi_lind_dt,e:dfi_lind_dnj,e:dfi_lind_de,e:dfi_troe_dt,e:dfi_troe_dnj,e:dfi_troe_de,e:dfi_sri_dt,e:dfi_sri_dnj,e:dfi_sri_de} may be represented in a general form via introduction of temporary variables $\Theta_{F_i, \partial \ldots}$:
\begin{align}
\lhs{dfi_gen}{\frac{\partial F_{i} }{\partial T }} &= \rhs{dfi_gen}{F_{i} \Theta_{F_i, \partial T}} & \cond{dfi_gen}{,} \\
\lhs{dfi_gen}{\frac{\partial F_{i} }{\partial {n_j}}} &= \rhs{dfi_gen}{\frac{F_{i} k_{0, i} \Theta_{F_i, \partial n_j}}{V k_{\infty, i}} \bar{\theta}_{P_{r, i}, \partial n_j}} & \cond{dfi_gen}{,}
\end{align}
and:
\begin{subequations}
 \begin{align}
  \lhs{dfi_gen}{\frac{\partial F_{i} }{\partial V }} &= \rhs{dfi_gen}{F_{i} \Theta_{F_i, \partial V}} & \cond{dfi_gen}{\dconp} \\
  \lhs{dfi_gen}{\frac{\partial F_{i} }{\partial P }} &= \rhs{dfi_gen}{F_{i} \Theta_{F_i, \partial P}} & \cond{dfi_gen}{\dconv}
 \end{align}
\end{subequations}
where for Lindemann falloff reactions:
\begin{align}
\label{e:dfi_lind_dt_prod}
\lhs{dfi_lind}{\Theta_{F_i, \partial T}} &= \rhs{dfi_lind}{0} \\
\label{e:dfi_lind_dnj_prod}
\lhs{dfi_lind}{\Theta_{F_i, \partial n_j}} &= \rhs{dfi_lind}{0}
\end{align}
and:
\begin{subequations}
 \label{e:dfi_lind_de_prod}
 \begin{align}
  \lhs{dfi_lind}{\Theta_{F_i, \partial V}} &= \rhs{dfi_lind}{0} \\
  \lhs{dfi_lind}{\Theta_{F_i, \partial P}} &= \rhs{dfi_lind}{0}
 \end{align}
\end{subequations}
For Troe falloff reactions:
\begin{align}
% troe
\label{e:dfi_troe_dt_prod}
\lhs{dfi_troe}{\Theta_{F_i, \partial T}} &= 
\rhs{dfi_troe}{ - \frac{B_{Troe}}{F_{cent} P_{r, i} \left(A_{Troe}^{2} + B_{Troe}^{2}\right)^{2} \log{\left (10 \right )}} \times \Biggl(} \nonumber \\
&\rhs{dfi_troe}{ 2 A_{Troe} F_{cent} \left(0.14 A_{Troe} + B_{Troe}\right) \times } \nonumber \\
&\rhs{dfi_troe}{\left( P_{r, i} \Theta_{P_{r,i}, \partial T} + \bar{\theta}_{P_{r, i}, \partial T}\right) \log{\left (F_{cent} \right )} + P_{r, i} \frac{\text{d} F_{cent} }{\text{d} T } \times \Bigl(} \nonumber \\
&\rhs{dfi_troe}{\mathindent 2 A_{Troe} \left(1.1762 A_{Troe} - 0.67 B_{Troe}\right) \log{\left (F_{cent} \right )} - } \nonumber \\
&\rhs{dfi_troe}{\mathindent B_{Troe} \left(A_{Troe}^{2} + B_{Troe}^{2}\right) \log{\left (10 \right )}\Bigr)\Biggr)} \\
\label{e:dfi_troe_dnj_prod}
\lhs{dfi_troe}{\Theta_{F_i, \partial n_j}} &= \rhs{dfi_troe}{- \frac{2 A_{Troe} B_{Troe} \left(0.14 A_{Troe} + B_{Troe}\right) \log{\left (F_{cent} \right )}}{P_{r, i} \left(A_{Troe}^{2} + B_{Troe}^{2}\right)^{2} \log{\left (10 \right )}}}
\end{align}
and:
\begin{subequations}
 \label{e:dfi_troe_de_prod}
 \begin{align}
   \lhs{dfi_troe}{\Theta_{F_i, \partial V}} &= \rhs{dfi_troe}{- \frac{2 A_{Troe} B_{Troe} \log{\left (F_{cent} \right )}}{P_{r, i} \left(A_{Troe}^{2} + B_{Troe}^{2}\right)^{2} \log{\left (10 \right )}} \times \bigl(} \nonumber \\
					    & \rhs{dfi_troe}{\mathindent[alittlebit] 0.14 A_{Troe} + B_{Troe}\bigr) \left(P_{r, i} \Theta_{P_{r,i}, \partial V} + \bar{\theta}_{P_{r, i}, \partial V}\right)} \\
   \lhs{dfi_troe}{\Theta_{F_i, \partial P}} &= \rhs{dfi_troe}{- \frac{2 A_{Troe} B_{Troe} \bar{\theta}_{P_{r, i}, \partial P} \left(0.14 A_{Troe} + B_{Troe}\right) \log{\left (F_{cent} \right )}}{P_{r, i} \left(A_{Troe}^{2} + B_{Troe}^{2}\right)^{2} \log{\left (10 \right )}}}
 \end{align}
\end{subequations}
where $\Theta_{P_{r,i}, \partial \ldots}$ and $\bar{\theta}_{P_{r, i}, \partial \ldots}$ are given by~\cref{e:dpr_theta_dt,e:dpr_theta_bar_dt,e:dpr_theta_bar_dnj,e:dpr_theta_dv,e:dpr_theta_bar_dv,e:dpr_theta_dp} respectively.
Finally, for SRI falloff reactions:
\begin{align}
 \label{e:dfi_sri_dt_prod}
 \lhs{dfi_sri}{\Theta_{F_i, \partial T}} &= 
 \rhs{dfi_sri}{- \frac{X \left(\frac{\operatorname{exp}\left({- \frac{T}{c}}\right)}{c} - \frac{a b}{T^{2}} \operatorname{exp}\left({- \frac{b}{T}}\right)\right)}{a \operatorname{exp}\left({- \frac{b}{T}}\right) + \operatorname{exp}\left({- \frac{T}{c}}\right)} + \frac{e}{T} - \Biggl(} \nonumber \\
 &\rhs{dfi_sri}{\mathindent \frac{2 X^{2} \log{\left (a \operatorname{exp}\left({- \frac{b}{T}}\right) + \operatorname{exp}\left({- \frac{T}{c}}\right) \right )}}{P_{r, i} \log^{2}{\left (10 \right )}} \times } \nonumber \\
 &\rhs{dfi_sri}{\mathindent \left(P_{r, i} \Theta_{P_{r,i}, \partial T} + \bar{\theta}_{P_{r, i}, \partial T}\right) \log{\left (P_{r, i} \right ) \Biggr)}} \\
 \label{e:dfi_sri_dnj_prod}
 \lhs{dfi_sri}{\Theta_{F_i, \partial n_j}} &= \rhs{dfi_sri}{- \frac{2 X^{2} \log{\left (a \operatorname{exp}\left({- \frac{b}{T}}\right) + \operatorname{exp}\left({- \frac{T}{c}}\right) \right )}}{P_{r, i} \log^{2}{\left (10 \right )}} \log{\left (P_{r, i} \right )}}
\end{align}
and:
\begin{subequations}
 \label{e:dfi_sri_de_prod}
 \begin{align}
  \lhs{dfi_sri}{\Theta_{F_i, \partial V}} &=
  \rhs{dfi_sri}{- \frac{2 X^{2} \log{\left (P_{r, i} \right )}}{P_{r, i} \log^{2}{\left (10 \right )}} \left(P_{r, i} \Theta_{P_{r,i}, \partial V} + \bar{\theta}_{P_{r, i}, \partial V}\right) \log \Biggl(} \nonumber \\
& \rhs{dfi_sri}{\mathindent \left( a \operatorname{exp}\left({\frac{T}{c}}\right) + \operatorname{exp}\left({\frac{b}{T}}\right)\right) \operatorname{exp}\left({- \frac{T}{c} - \frac{b}{T}}\right) \Biggr)} \\
  \lhs{dfi_sri}{\Theta_{F_i, \partial P}} &=
  \rhs{dfi_sri}{- \frac{2 X^{2} \bar{\theta}_{P_{r, i}, \partial P} \log{\left (P_{r, i} \right )}}{P_{r, i} \log^{2}{\left (10 \right )}} \log{\left (a \operatorname{exp}\left({- \frac{b}{T}}\right) + \operatorname{exp}\left({- \frac{T}{c}}\right) \right )}}
 \end{align}
\end{subequations}

\subsubsection{Unimolecular\slash~recombination fall-off reactions (temporary product form)}
Using the temporary products developed in~\cref{s:dpr,s:dfi}, the fall-off reaction derivatives developed in~\cref{e:dci_fall_dt,e:dci_fall_dnj,e:dci_fall_de} may be simplified to:
\begin{align}
\lhs{dci_fall_complete}{\frac{\partial c }{\partial T }_{i}} &= \rhs{dci_fall_complete}{\frac{F_{i} \bar{\theta}_{P_{r, i}, \partial T}}{P_{r, i} + 1} + \biggl(- \frac{P_{r, i} \Theta_{P_{r,i}, \partial T}}{P_{r, i} + 1} + \Theta_{F_i, \partial T} + } & \nonumber \\
							     &  \rhs{dci_fall_complete}{\mathindent[largerindentpleas] \Theta_{P_{r,i}, \partial T} - \frac{\bar{\theta}_{P_{r, i}, \partial T}}{P_{r, i} + 1}\biggr) c_{i}} & \cond{dci_fall_complete}{,} \\
\lhs{dci_fall_complete}{\frac{\partial c }{\partial {n_j} }_{i}} &= \rhs{dci_fall_complete}{\frac{k_{0, i} \bar{\theta}_{P_{r, i}, \partial n_j}}{V k_{\infty, i} \left(P_{r, i} + 1\right)} \left(F_{i} \left(P_{r, i} \Theta_{F_i, \partial n_j} + 1\right) - c_{i}\right)} & \cond{dci_fall_complete}{,}
\end{align}
and:
\begin{subequations}
 \begin{align}
  \lhs{dci_fall_complete}{\frac{\partial c }{\partial V }_{i}} &= \rhs{dci_fall_complete}{\frac{F_{i} \bar{\theta}_{P_{r, i}, \partial V}}{P_{r, i} + 1} + \biggl(- \frac{P_{r, i} \Theta_{P_{r,i}, \partial V}}{P_{r, i} + 1} + \Theta_{F_i, \partial V} + } & \nonumber \\
							       &  \rhs{dci_fall_complete}{\mathindent[largerindentpleas] \Theta_{P_{r,i}, \partial V} - \frac{\bar{\theta}_{P_{r, i}, \partial V}}{P_{r, i} + 1} \biggr) c_{i}} & \cond{dci_fall_complete}{\dconp} \\
  \lhs{dci_fall_complete}{\frac{\partial c }{\partial P }_{i}} &= \rhs{dci_fall_complete}{\frac{F_{i} \bar{\theta}_{P_{r, i}, \partial P}}{P_{r, i} + 1} + \left(\Theta_{F_i, \partial P} - \frac{\bar{\theta}_{P_{r, i}, \partial P}}{P_{r, i} + 1}\right) c_{i}} & \cond{dci_fall_complete}{\dconv}
 \end{align}
\end{subequations}
where the temporary products are given by~\cref{e:dpr_theta_dt,e:dpr_theta_bar_dt,e:dpr_theta_bar_dnj,e:dpr_theta_dv,e:dpr_theta_bar_dv,e:dpr_theta_dp}~and~\cref{e:dfi_lind_dt_prod,e:dfi_lind_dnj_prod,e:dfi_lind_de_prod,e:dfi_troe_dt_prod,e:dfi_troe_dnj_prod,e:dfi_troe_de_prod,e:dfi_sri_dt_prod,e:dfi_sri_dnj_prod,e:dfi_sri_de_prod}

\subsubsection{Chemically-activated bimolecular reactions (temporary product form)}
Similarly, the chemically-activated reaction derivatives in~\cref{e:dci_chem_dt,e:dci_chem_dnj,e:dci_chem_de} may be written as:
\begin{align}
\lhs{dci_chem_complete}{\frac{\partial c }{\partial T }_{i}} &= \rhs{dci_chem_complete}{\left(- \frac{P_{r, i} \Theta_{P_{r,i}, \partial T}}{P_{r, i} + 1} + \Theta_{F_i, \partial T} - \frac{\bar{\theta}_{P_{r, i}, \partial T}}{P_{r, i} + 1}\right) c_{i}} & \cond{dci_chem_complete}{,} \\
\lhs{dci_chem_complete}{\frac{\partial c }{\partial {n_j} }_{i}} &= \rhs{dci_chem_complete}{\frac{k_{0, i} \bar{\theta}_{P_{r, i}, \partial n_j} \left(F_{i} \Theta_{F_i, \partial n_j} - c_{i}\right)}{k_{\infty, i} V \left(P_{r, i} + 1\right)}} & \cond{dci_chem_complete}{,}
\end{align}
and:
\begin{subequations}
 \begin{align}
  \lhs{dci_chem_complete}{\frac{\partial c }{\partial V }_{i}} &= \rhs{dci_fall_complete}{\left(- \frac{P_{r, i} \Theta_{P_{r,i}, \partial V}}{P_{r, i} + 1} + \Theta_{F_i, \partial V} - \frac{\bar{\theta}_{P_{r, i}, \partial V}}{P_{r, i} + 1}\right) c_{i}} & \cond{dci_chem_complete}{\dconp} \\
  \lhs{dci_chem_complete}{\frac{\partial c }{\partial P }_{i}} &= \rhs{dci_fall_complete}{\left(\Theta_{F_i, \partial P} - \frac{\bar{\theta}_{P_{r, i}, \partial P}}{P_{r, i} + 1}\right) c_{i}} & \cond{dci_chem_complete}{\dconv}
 \end{align}
\end{subequations}

\subsection{Final jacobian form}
\label{s:jac_final}


\bibliography{derivations}

\end{document}
